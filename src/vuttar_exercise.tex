\articlepart{タイピング練習論}{テル(vuttar)}

\section{イントロダクション}

\subsection{はじめに}

「どうすればタイピングが速くなるのか?」

競技タイパーにとっては最も身近で切実な、常に向き合っていかなければならない問題です。しかし残念ながら「これこれこうすれば絶対にタイピングが速くなりますよ!」という、どんな人にも当てはまる夢のような指針は今は存在せず、出来上がる気配もありません。各個人により現時点での実力や打鍵スタイル、求める「速さ」の形が異なるなど、様々な問題があり、結局「とにかく練習しろ」などと半ば投げ捨てられることが多いように思います。

しかし、競技タイパーとしてはこの問いから逃れることはできません。自分が求める「速さ」とは何なのか、その「速さ」を伸ばすためにはどんな練習をすればいいのか、タイパー各個人が常に考えていく必要があります。とは言っても、一人一人の考えには限界があります。実力や打鍵スタイル、求める「速さ」はそれぞれ違っても、積極的に意見交換をして、考えを洗練させていくことが必要だと思っています。

この記事では、私自身や皆さんにとってタイピングの上達方法に関する意見交換の助けとなるよう、普段私が考えていること、実践していることをまとめていきます。また、上達方法を語る上で整理しなくてはならない問題がいくつかあるため、そういった様々な問題にも触れていきます。

\subsection{注意点}

先程「一人一人の考えには限界がある」と書きましたが、その最たる例がタイピングスタイルの違いです。使用する配列、最適化の有無・程度、変則運指ベース・標準運指ベース、正確性派・最高速度派などのスタイルの違いにより、それぞれのタイピングは大きく変わります。特に運指ベース、最適化の有無が重大な違いになると思います。スタイルが変われば当然上達方法も変わってくるため、タイピング全般に通用する議論をしようと思うと、そういった違いに配慮する必要が出てきます。

とはいえ、そこまで考えて話を進めていくのは大変ですし、そもそも私自身にその能力がありません。そこで、今回は基本的に私のタイピングスタイル(文字列パターン別の担当指最適化の不採用、タイプウェル重視、正確性・安定性重視、標準的な運指ベース等々)を前提として話を進めていこうと思います。つまり内容的には「あくまで私の場合はこうである」という程度のものになってしまうわけですが、私のタイピングスタイルが比較的オーソドックスなこともあり、注意深く読んでいただければ、他の様々なタイピングスタイルにも十分適応できると思っています。

\subsection{この記事の対象者}

この記事は、ある程度タイピングに習熟している方や、成長の壁を感じている方を主な対象としています。タイピングを始めたばかりの方や、全く壁に直面せず順調に成長している人は、あまり深く考えず、とにかくがむしゃらに練習していくのが良いと思っています。また、確固たる根拠や理論ではなく、単なる経験や感覚に基づいて話を組み立てていくので、皆さん自身の経験や考察に基づいて批判的に読んでほしいという理由もあります。始めたばかりの方で「それでも内容が気になる!」という場合、2~5章はとりあえずさらっと読み流して、具体的な練習方法に言及している6, 7章をお読みいただくことをお勧めします。

\subsubsection*{習熟の目安}
\begin{itemize}
 \item 単キー打鍵は無意識化している
 \item 文字をまとまりで捉えてよどみなく入力できる
 \item 先読みをある程度意識できる
 \item とにかく打っていれば楽に更新できる、という時期を過ぎた
 \item つまるところ「認識・組立・動作を、並行してある程度よどみなく行える」ということ
 \item 目安としては タイプウェル国語R 常用 XF ぐらいから?
\end{itemize}

\subsection{「速さ」とは何か}

「どうすればタイピングが速くなるのか?」という問いは、単純明快のようでいて、よくよく考えてみると意味が分かりません。タイピングが「速い」とはどういうことか、という問題に答えが出ていないからです。例えばタイプウェル国語R基本常用語の記録が伸びたとして、それで「タイピングが速くなった」と言えるのか、といった問題です。どれか一つのソフト(例えばタイプウェル)で良い記録を出せても、他のソフト(例えば e-typing 、毎パソ、Weather Typing ……)で同じレベルの記録を出せるとは限りません。つまり、各ソフトの「速さ」は似て非なるものなのです。

各ソフトの「速さ」がそれぞれ異なるとすれば、一体何がどう違うのか、ということが問題になってきます。私個人の考えとしては、出題文の内容や打鍵数、出題の仕方、受付入力方式の差異など、各ソフトの特性に違いがあり、その結果として「各ソフトが要求する能力」が大きく違ってくることが問題だと考えています。「速さ」を示す数値(= 「打鍵速度」「 kpm 」「打/秒」)は同じであっても、それぞれのソフトで同じ数値に達するために各ソフトが要求する能力が違えば、それぞれの数値は別物と言えるだろう、ということです。

例えば「毎回同じ文章が出題されるソフト(毎パソ、歌謡タイピング劇場、タイプウェル憲法など)」では、効率の良い運指やペース配分を事前に頭の中で組み立てておき、何度も練習して定着させ、指の運動能力を最大限に活用する、といった能力が重要になるかもしれません。「決まった単語・文章の中からランダムで出題されるソフト(タイプウェル、e-typing 、寿司打など)」では、その時々で運指やペース配分を考えねばならず、運動能力よりは脳内での処理能力が重要になるかもしれません。

また、出題文字数によっても「速さ」を得るために必要な能力は大きく変わってきます。数千打を要する毎パソでは持久力、400打固定のタイプウェル(国語R, E)では最高速を持続する能力、短文の e-typing では研ぎ澄まされた集中力が重要になるかもしれません。

このような各ソフトの特徴の違いを全て考慮して、各ソフトがどのような能力をどの程度要求するかを割り出し、更にどのような能力をどのような配分で持っていることが理想であるかの合意に至れば、各ソフトの成績を総合して「絶対的な速さ」を競うことができるかもしれません。しかし現実には、競技毎の特徴はある程度分かったとしても(実際は競技特性自体の分析もまだ進んでいないとは思いますが)、それぞれの特徴を有したタイピング競技がどのような能力を必要とするのかがよく分かっておらず、残念ながら絶対的な速さを計測することは不可能です。また同時に、ある競技の記録を伸ばすにはどのような練習をすればいいかとか、自分が求める「速さ」を高めるためにはどういう練習をすればいいか、といった問いにも完全な答えは出ないということになります。

しかし、競技特性が要求する能力の違いを全く考慮せずに練習するというスタンスは、あまり好ましいものではないでしょう。自分がどういう「速さ」を得たいのか、どういう能力を要求する競技を重要視するのか、という展望はタイピングを深く楽しむために重要なものだと思いますし、ある特定の競技の記録向上のみを考えるにしても、その競技が要求する能力を吟味することは(例えイマイチ分からなかったとしても)効率的な成長を遂げるために重要なはずだからです。また、こういった問題についての理解が深まらないと、記録の意味を正しく捉えることができません。他人の記録、さらに自分の記録を尊重するためにも、皆で考えていかなければならない問題だと思っています。

\subsection{現実的な妥協点}

ということで、この問題を多少なりとも解決するため、私なりの「各競技が要求する能力を区別する」ための方法と、その上で今回の記事がどういう「速さ」を主眼におくのかを示し、それに基づいた練習論を展開しようと思います。

まず、「各競技が要求する能力」の前段階の、「そもそもタイピングとはどういう能力の組み合わせで成り立っているのか」という問いの更なる前段階として、「タイピングとはどのような過程で行われるのか」ということを考えます。現実的には「これこれこの器官が文字列を光として受容して脳がこう働いて電気信号がこうで筋肉がこう反応して指がこう動いて……」のように、無限にも近い過程が組み合わさってタイピングが進行するわけですが、それを大雑把に「認識」「組立」「動作」の三つに分類してしまいます。この三つの過程に分類すれば、それぞれの過程でどのような能力が必要になるか、といったことが多少分かりやすくなるだろう、という魂胆です。この考えを2, 3章で詳しく説明し、4章では様々な競技にこの考えを適用して、競技の特性を分析します。

そうしてタイピング能力をある程度相対化した後は、全ての競技特性に通用する形で議論を進めるのは私には不可能なので、ほぼタイプウェルのみに話を絞ります。「タイプウェルがどのような能力を要求するか」をトライアル中のフェーズ毎に分析(5章)し、「タイプウェルが要求する『速さ』」を他の競技と一応比較できるよう相対化した上で、タイプウェルの練習方法を語る(他の競技への適用は皆さんに考えてもらう)(6章)ということです。これが今回の記事の妥協点というか、私の限界ということになります。また、7章ではこういった枠組みは脱して、どのような競技にも共通する、コンディションの調整・長期的な練習計画・成長のビジョン・モチベーションの維持、といった一般的な話題について語ります。

\section{タイピングの三要素}

\subsection{タイピングを分解する}

前章で説明した通り、タイピングは数えきれない段階的な処理から成り立っています。これらの処理を細かく分解し、それぞれの処理毎に時間を短縮するための対策を考えていけば、より効率的に上達できるでしょう。しかし、細かく分ければ分けるほど負担は重くなりますし、細部にこだわりすぎて全体が見えなくなってしまうということも考えられます。そこで、まずは大雑把に2~5個程度の要素に分解してみると、適度に細かい部分まで目が行くように思います。私の場合は、3つの要素に分解します。
\subsection{タイピングの三要素}

この記事では、タイピングの段階的な処理を、大きく「認識」「組立」「動作」の三つに分解して考えます。「認識」は画面に表示された課題文字列を認識・記憶すること。「組立」は認識した文字列をもとに脳内で打鍵動作を組み立てること。「動作」は脳内で組み立てた打鍵動作をもとに、実際に指を動かすことです。この章では、この三要素についての説明をします。また、三要素の相互関係については3章以降で考えます。

\subsection{認識}

画面に映し出された映像をとらえ、課題文字列として認識し、記憶する処理のことです。例えば「馬の耳に念仏」という課題文字列が表示された時、「馬・の・耳・に・念・仏」と認識するか、「馬の・耳に・念仏」と認識するか、「馬の耳に念仏」と認識するかで、認識速度は大きく変わってきます。先読み(既に認識した課題文字列の打鍵動作が終わる前に、先へ先へと課題文字列の認識を進めていくこと)と、先読みした課題文字列を短期的に記憶することも「認識」の範囲に入れることとします。

\subsection{組立}

認識した課題文字列を打鍵列\footnote{打鍵列とは課題文字列とは違い、実際に押下するキーの並びのことです。「孝行」という課題文字列に対し、打鍵列は「koukou」「coucou」「koucou」など複数に分かれる場合もあります。かな入力であれば打鍵列は「こうこう」となります。}に変換し、その打鍵列に対応する打鍵動作を脳内で組立てる(イメージする)処理のことです。パターン別の担当指最適化を行なっている人の場合、使用する指の選択なども組立の範囲に入ります。打鍵列への変換は軽視されがちのように思いますが、重要な処理です。

\subsection{動作}

脳内で組立てた打鍵動作をもとに、実際に指を動かす処理のことです。よくよく考えると、組立と動作の境目は非常に曖昧なのですが、この記事ではあまり深く考えないことにします。

\subsection{ミスについて}

タイプミス後の打ち直しや修正(毎パソなどの実用入力では必要になる)についても、通常の打鍵と同様に「ミスの認識」「打ち直し・修正動作の組立」「実際の打ち直し・修正動作」と、タイピングの三要素に分けて考えることができます。ただ、ミスの認識・打ち直し・修正などは、本来は別枠で考えなければならない問題のように思いますので、今回は深く考えないことにします。

\subsection{処理の順番?}

この考え方のもとでは、タイピングは「認識→組立→動作」という過程を経て行われる、ととらえます。しかし、別に「認識→組立→動作→認識→組立→動作→認識→……」という順番で行われる、というわけではありません。タイプウェル国語R 基本常用語で「あらかじめ_まずまず_おぼろげ_」という課題文字列があった場合、「あらかじめ」を認識し、「arakajime」と変換し、打鍵動作を組立て、キーを押下し終えてから次の「まずまず」を認識し、「mazumazu」と変換し……、というわけではないのです。実際には、「arakajime」の部分の動作をしながら、「mazumazu」の部分を組立て、同時に「おぼろげ_」の部分を認識するといったように、それぞれの動作は並行しています。三要素は基本的には並行しながら、様々に関係し合います。この関係について次の章で考えます。

\section{三要素の相互関係}

\subsection{三要素の相互関係}

実際のタイピングの局面では、「認識」「組立」「動作」の三要素は独立に進行するわけではなく、複雑に関係しています。つまり、「認識速度が700kpm(に相当する量の文字数/分)、組立速度が600kpm、動作速度が650kpmだから、タイピング速度は一番低い組立速度に足を引っ張られ、600kpmとなる」というような単純なものではありません。ある時は認識が足を引っ張り、またある時は動作が足を引っ張り……というように、打鍵速度のネックとなる部分は時と場合(出題文字列、身体や精神の状態、プレイしているソフトの特性など)によって変わります。タイパーはよく「現時点では認識が足を引っ張っている」とか「指の動作速度がネックになってきた」といったことを言いますが、これは「実際の局面では足を引っ張る部分は様々に変わるけれども、特に認識速度が大きく足を引っ張るようになってきた」ということである、という点に注意しなくてはなりません。

\subsection{三つのキーワード}

では、具体的には三要素はどのような相互関係にあるのでしょうか。これを考える際のキーワードは、「並行」「休止」「減速」だと思っています。並行とは三要素が同時進行している通常の状態、休止とは三要素のどれかが働いていない状態、「減速」とは三要素のどれかが減速している状態のことを言います。

\subsection{並行}

ある程度まで習熟したタイパーであれば、タイピングの多くの場面では「認識」「組立」「動作」を全て同時に行なっている(並行している)状態となります。2章で例に挙げたように、「あらかじめ_まずまず_おぼろげ_」という課題文字列のうち、「arakajime」の部分の動作をしながら、「mazumazu」の部分を組立て、同時に「おぼろげ_」の部分を認識していく、といったような状態です。このように三要素が「並行」している状態を、通常の状態としてとらえます。

\subsection{休止}

しかし、実際のタイピングの局面には三要素が並行していない状態もあります。三要素の一部が働かなくなることを、「休止」と呼ぶことにします。それぞれのタイピングの局面でどの要素が休止して、逆にどの要素が働いているのか、よく考えることが重要です。

休止の例としては、タイプウェルの最初と最後が非常に分かりやすいです。タイプウェル国語Rの最初で「(開始)あらかじめ_まずまず_」という課題文字列が表示されたとします。タイパーは一瞬で「あ、ら、か、じ、め、『あらかじめ』だな」と認識しますが、この時はまだ組立・動作は休止しています。このあと組立が始動し、最後に動作が始動します。また、タイプウェル国語Rの最後で「_おぼろげ_さすらう(終了)」という課題文字列が表示されたとします。それまで三要素は全て並行していますが、「_さすらう」まで読み終わった段階で認識が休止し、そのあと組立が休止します。終了直前の一瞬は、動作だけが働いている状態となります。

他にも要素が休止する例はいくらでも挙げられます。例えば組立・動作に手間取りすぎて先読みを中断した場合(認識の休止)、ミスをしてしまいミス内容を確認している場合(組立と動作の休止)、完全に打ち慣れた課題文字列が表示され、一瞬で認識と組立を終えた場合(認識と組立の休止)などがあります。

\subsection{減速}

認識も組立も動作も、常に最高速で展開されるわけではありません。状況次第で、どれかの要素、もしくは全ての要素が減速を強いられます。いくつか減速の具体例を挙げていこうと思います。

\subsubsection*{単純に難しいパターン}

認識・組立・動作の全てにおいて、処理するのが難しいパターンが存在します。難しいパターンに直面すれば、当然それぞれの処理は減速します。漢字モードで難しい漢字が出てくると認識が、「わざわざ」「牛乳」といった難パターンが出てくると動作が、スペル打ち分けをする人は打ち分け対象の文字が出てくると組立が減速する、といった具合です(例えば \key{C} と \key{K} の打ち分けをしている人は、「こ」が出てくると組立が減速します)。

\subsubsection*{他の要素との兼ね合い}

他の要素との兼ね合いによっても、それぞれの処理速度は大きく変わってきます。例えば「_わざわざ_うんぬん_」といった打ちにくい文字列が見えた時、動作速度が遅くなるのを見越して認識を減速する場合があります。打ち慣れない文字列が出てきて組立が減速し、それに伴って動作も減速し、その遅れを調整するために認識を減速する、と連鎖していく場合もあります。認識・組立・動作のどれかが物凄く苦手で、ほぼ常に他の処理を減速させている場合も考えられます。

タイピング上達の基本は、この「三要素の兼ね合い」をよく見定めることだと思っています。例えば先読みが足りず詰まってしまった場合、先読みする量を増やす(認識を改善する)ことで対応しよう、とただちに考えるのは早計です。これがもし「難しい動作パターンに対応するために認識が減速し、その直後の認識の立て直し(認識の再加速)ができずに詰まってしまった」というケースであれば、改善すべきはむしろ動作能力、もしくは認識の再加速能力、ということになるはずです。個別の局面について常にこういう考察を入れていくことは難しいですが、基本的に、三要素を常にバランス良く働かせていく、という意識を持つことが重要だと思います。

\subsubsection*{意識的・無意識的なペース調整}

認識・組立・動作の全てに余裕がある、つまり全ての要素を無駄に減速させてしまっている、という場合も考えられます。これまで「三要素の速度が関係し合ってタイピング速度を決めている」という書き方をしてきましたが、それぞれの能力を限界まで駆使することは難しいので、実際には、練習によって積み重ねた「このくらいのペースでなら打てる」というペース感覚によって、基本的な速度を制御しているはずです。このペース調整は意識的に行う場合が多いですが、無意識的に行われている場合もあります。「このペースを越えると打てない」と思い込んでしまい、本来の能力的にはもっと速度を上げられるはずなのに、速度を抑えてしまう場合です。一度記録を大幅更新すると「あれ、打てるじゃん」という感じになって同レベルの記録を連発できたり、一定期間タイピングをやめていて久しぶりにやり直したら大幅更新できてしまった、という場合などは、これを乗り越えた一例だと思っています。大幅更新で一段階速いペース感覚を手に入れたことや、一定期間の放置によってペース感覚がリセットされたことにより、本来の能力が発揮されたのではないか、という解釈です。

\subsubsection*{コンディションと外部的要因}

集中力が落ちてくれば認識・組立の能力は低下しますし、筋肉に疲労がたまってきたり、寒さによって指がかじかんでいれば、動作能力は低下します。こういった「コンディション」は非常に重要です。コンディションの調整については7章で扱います。また、課題文の文字サイズやフォント、背景色などによって認識が減速するなど、外部的な減速要因も考えられます。こういった外部的要因がタイピングの邪魔をしていないかということも、たまには考えてみる必要があるでしょう。

\subsection{まとめ}

認識・組立・動作の三要素は、通常の状態では並行して働いています。しかし、タイピングの局面によっては一部の要素が休止したり、減速することがあります。ある要素の減速に伴って他の要素が連鎖的に減速したり、ある要素の減速を見越して意識的に他の要素を減速させたり、ということもあります。逆に言えば、ある要素の能力を改善することによって、他の要素での減速を防ぐことができます。つまり、認識・組立・動作の三要素の基本的な相互関係は、どれかに手間取れば他の要素にも悪影響を及ぼし、どれかが改善すれば他の要素にも多少の好影響を及ぼす、というものだと考えられます。ただし相互関係の具体的なあり方は、それぞれの競技のそれぞれの局面、その競技に対する熟達度、タイピングスタイルの違いなどによって大きく変わってきます。そういった違いについては、4章、5章で考えていきます。

\section{三要素の考えを各ソフトに適用}

\subsection{おことわり}

この章では、三要素の考えをいくつかのソフトに適用して、それぞれのソフトが各局面で認識・組立・動作をどのように要求するかを考えます。この試みは、それぞれのソフトが要求する能力を分類して、ある単一のソフトでの記録や成果を他のソフトの記録や成果と対比しやすくしたり、練習計画の決定に役立てるためのものです。ただし、これらのソフトは私自身があまりやり込んでいないこともあり、考察の精度にはかなり疑問が残ります。三要素を使った競技特性の分類法の一例を紹介した、という程度に受け取っていただければ助かります。また、この章を読んで「いや、このソフトの本質はそんなもんじゃない!」と思ったガチ勢の方に批判・再分析していただければ非常に有難いです。

\subsection{e-typing 腕試しレベルチェック}

e-typing で最も人気のある形式です。大体6~30打鍵相当程度の課題文が表示され、それを打ち込みます。課題文は、定期的に変更される課題文セットの中から、1トライアルにつき15回出題されます。一つの課題文を打ち終わってから次の課題文が表示されるまでには少し間があるため、1文ごとの独立性が非常に高く、実質的には15回のトライアルの組み合わせからなる競技と解釈していいと思います。どのソフトにも言えることですが、このソフトでは特に習熟度(≒ワード慣れ)による要求能力の変化が大きいので注意が必要です。

まず、e-typing を始めたばかり~それなり程度にしかワード慣れしていない人の場合を考えます。3章 4節「休止」の部分で触れましたが、課題文が出題された直後は、認識・組立のみを行っている時間が存在します。本格的に動作を始めるには、少なくとも5~6打鍵分は認識・組立を進める必要があるため、この影響は無視できません。更にその「打ち始めの部分」が15回あるため、動作が休止している時間、つまり認識・組立のみを要求する時間が他のソフトと比べて非常に長く、打ち始めの認識・組立の速さが記録を大きく左右すると考えられます。また、同様に「打ち終わりの部分」も15回存在します。打ち終わりの部分では逆に動作のみが働く(認識・組立は既に終わっている)こととなるため、動作能力を限界まで使ってスパートをかけることで、時間を短縮することができます。しかし課題文の文字数が少ないため、打ち終わりまでに動作の限界速度に到達することは難しく、動作速度を素早く高める能力(初期段階での動作加速力)が重要となってきます。

次に、e-typing の特定の課題文セットをやり込み、極度にワード慣れした人の場合を考えます。まず認識は、ひらがな換算で数文字を読むか、全体をざっと見るだけで終了する(課題文を覚えているので、軽く確認すれば全文を思い起こせる)はずです。また、組立については更に顕著で、課題文認識の後、長い練習によって積み重ねた打鍵感覚を呼び起こすだけで、組立が終了するはずです。保存しておいた打鍵動作を解凍する、とイメージすると分かりやすいと思います。このため、ワード慣れしていない人の e-typing では認識・組立の速さが非常に重要になっていたのに対して、極度にワード慣れした人の場合、認識に関しては覚えたワードを即座に引き出す速さ、組立に関しては積み重ねた打鍵感覚の精度とそれを引き出す速さ、だけが重要になってくると考えられます。認識と組立が即座に済んだ後は、ひたすらに動作の速度と精度が要求されることになります。また、課題文が短いことが多いので、純粋な動作能力に加え、初期段階での動作加速力も非常に重要となってくると思います。このように、極度にワード慣れした人の場合、e-typing で要求される能力としては、圧倒的に動作能力の比重が高くなります。

ワード慣れしていない人、している人の場合をそれぞれ考察しましたが、実際は「全くワード慣れしていない」「極度にワード慣れしている」ということは中々無いはずですので、課題文ごとのワード慣れの程度によって、上記二つの間を行き来することになると思います。具体的には、ワード慣れしていなければいないほど認識・組立の能力が、ワード慣れしていればしているほど動作の能力が重要になってくる、ということになるでしょう。また、初期段階での動作加速力は、ワード慣れの程度に関わらず重要になってくると思います(あるいはこれこそが e-typing の要点かもしれませんが、私にはまだ確かなことは分かりません)。

\subsection{Weather Typing}

Weather Typiing (以下 WT )はモルタルコ氏(本誌でのインタビューにも答えてくださっています。そちらもあわせてご覧ください)が制作・公開されているフリー(無料公開)のタイピングソフトです。WT ではワードセットを自作することもできますが、今回はデフォルトのワードセット(具体的には word1.txt )の使用を前提にします。WT のシステムは独特で、ワードセット(前半)とワードセット(後半)が分かれていて、それぞれをランダムに組み合わせて、実際に出題される課題文が決定されます。例えば「引き出しの中に悪の権化」「暖かいカーテンコール」など、多様な組み合わせの課題文が出題されます。このゲームの場合、ため打ち無し(主に通信対戦。課題文を確認次第すぐに打つ)とため打ちあり(主に一人プレイ。課題文を確認した後、脳内で念入りに打鍵動作を組み立ててから最高速で打ち抜く)で大きく競技特性が変化します。

まずため打ち無しの通信対戦を想定します。この場合は、e-typing に非常に近い特徴を持ちます。WT では「課題文表示から打ち始めまでの時間」はカウントされないため、認識・組立の重要性は本来低いのですが、対戦で悠長にやっているとワードを取られてしまうため、実質的には認識・組立の能力が重要となります。また、e-typing に比べると課題文が比較的長いことが多いので、初期的な動作加速力の重要性は若干低くなります。ただし、問題文が長くなる分、初期的な加速を終えた段階で打ち終わりを迎えられるので、打ち終わり付近(認識・組立が休止している部分)の速度に動作能力の高さが大きく影響します。

次にため打ちありの一人プレイを想定します。この場合、ため打ち無しとは競技としての特徴が非常に大きく異なります。まず、ミスの確認や進行状況の確認を除けば、認識能力が全く要求されません。どれだけじっくりと問題文を読んでも、その時間はカウントされないからです。組立の速度も同じく要求されません。ただし、組立の精度(これを動作の精度と厳密に区別するのは難しいのですが、それについては無視します)は要求されます。打鍵動作をしっかり組み立てたつもりでも、実は細部が曖昧になっていて、そのせいで動作が止まってしまう、というのはよくある話です。また、動作加速力の影響も非常に少なくなります。実際に打ってみると分かると思いますが、事前に組立を念入りに行うことで、打ち始めてすぐにほぼ最高速まで加速することができます。つまりため打ちありの WT においては、若干の組立の精度の他は、ほとんど純粋な動作速度(≠動作加速力)のみが要求される、と考えられます。

\subsection{タイプウェル憲法}

タイプウェル憲法はタイプウェルシリーズ( GANGAS 氏が制作・公開されているタイピングソフト。やはりインタビューに答えてくださっているので、そちらもあわせてご覧ください)の一つで、超長文の「通し」と、短文~長文の「条項別」の二つのモードがあります。この二つのモードについて、別々に考えます。話を簡単にするため、通しモードについてはワード慣れはあまりしておらず、条項別についてはある程度ワード慣れしている、と仮定します。通しモードは非常に長いため全文に慣れることは難しく、条項別については短時間に同じ条項を打ち込むことで簡単にワード慣れできるため、おおむね問題のない仮定だと思います。

\subsubsection*{通しモード}

打ち始めについては、何度か打っていれば各章の始めから十数文字くらいまでは覚えられるので、認識能力はほぼ要求されません。同様に打ち始めの十文字程度までは事前に打鍵動作を組み立てておけばいいので、組立能力もほぼ要求されません。したがって WT のため打ちプレイと同様に、打ち始めはほぼ動作速度のみが要求されると考えられます。その後は安定的に三要素が並行し続けるはずですが、もちろん、3章で述べたような様々な要因で変化することはあります。特にタイプウェル憲法では、日常の文章では目にしない文体や漢字の読みが頻発するため、認識面で減速しやすいです。また、条項が変わる部分では、先読みのために視線を動かす必要があり、認識面での大きな負担から、認識・組立・動作の全てが圧迫されます。この対策としては、あらかじめ組立・動作を遅らせて認識とのギャップを減らすことや、意識的に先読み分を増やして認識の余裕を作ること(これは非常に難しいような気もします。私はやっていません)などが挙げられます。また、後半になってくると集中力・筋持久力的な問題が出てきます。筋持久力の問題は動作への圧力と考えられ、常に無駄のない打鍵を行う動作能力(動作の洗練性)や、疲労の程度に応じて打ち方を変えるような柔軟な組立能力が求められます。打ち終わりの数文字~十数文字は他の競技と同じように動作速度のみが要求されますが、筋持久力的な限界が来ているであろうこと、全体の中での割合が極端に小さいことを考えると、あまり重要ではないでしょう。

\subsubsection*{条項別モード}

打ち始めについては、通しモードと同じくほぼ動作速度のみが要求されます。条項別の記録を狙う際は基本的に同じ条項を連続して打つため、課題文についてはかなりの部分まで覚えることができ、打鍵動作も感覚として保てるため、やはりほぼ組立の精度・動作速度のみが要求されます。ただし、ある程度長い条項の場合は、課題文を覚え切れず、認識と組立の能力も要求されるようになります。また、最高速に近い状態である程度長い間打鍵する必要が出てくるため、筋持久力・集中力ともに早いペースで消費されます。ただ分量的にはゴリ押しで乗り切れる程度とも思えますので、疲労状態にあっても認識・組立・動作の精度と速度を失わないことが重要になってくるのかもしれません。

\subsection{まとめ}

e-typing 、WT 、タイプウェル憲法について、やや表面的にではありますが、三要素がどのように要求されるかを考察してみました。内容の妥当性には疑問が残りますが、こういった考察を通してそれぞれの競技がどのような能力を要求するかを大雑把に分析することができる、ということは分かっていただけると思います。その分析の結果と自分の現在の能力を照らし合わせ、どの能力をどう向上していけば効果的なのか、ということを考えて練習していくことが重要です。

\section{三要素の考えをタイプウェルに適用}

\subsection{おことわり}

この章では、タイプウェル国語R 基本常用語 を例に挙げ、競技の各局面でどのような能力が要求されるかについて、三要素の考えを用いて、前章よりは少し丁寧に考察します。それぞれの局面で必要とされる能力を踏まえて、具体的にどのようなことに気を付けるべきか、といったことについても触れていきます。国語R 内の他のモードについては、最後に簡単に触れます。

\subsection{タイプウェル固有の要素}

\subsubsection*{スペース入力}

単語と単語の間で \key{Space} の入力を要することは、タイプウェルの大きな特徴です。ある程度慣れさえすれば、\key{Space} 入力の存在は、認識・組立・動作の全てに好影響を及ぼすことと思います。ただし、本当に重要なのは、それぞれの要素に及ぼす影響の“程度”です。\key{Space} が打ちやすいと言ってもある程度の負担はあるはずですので、それぞれの要素にどの程度の負担があるか考えてみましょう。

実際に \key{Space} の押下をする必要がありますので、動作については確実に一定の負担があります。組立についても、同じ「\key{Space}を押す」という打鍵動作でも、実際の状況次第で押し方は変わるため、毎回打鍵動作を組み立てる必要があり、多少の負担はあるはずです。しかし認識については、\key{Space} の存在は毎回分かり切っていることで、それぞれの単語だけ認識すれば済むので、何の負担にもなりません。よって、基本的に \key{Space} 入力の存在は「認識面の負担が相対的に軽くなる」という特徴をもたらすと考えられます。

\subsubsection*{課題文字列の三表記}

タイプウェル国語Rの課題文字列表記は、通常、上段の漢字平仮名混じり表記・下段のローマ字表記・下段の流れる文字表記 の3つに分かれています。これらをどう使うかによって、認識面への負担が変わってきます。常に自分に合ったやり方を模索することが重要です。特に認識面がネックになっていると感じている場合には、まず改善を検討すべき部分だと思います。

\subsubsection*{単語単位でのランダム出題}

タイプウェル国語Rの課題文字列出題形式は、タイプウェル憲法のような固定長文でも、e-typing のような文単位でのランダム出題でもなく、単語単位でのランダム出題です。\key{Space} による区切りはあるものの、主に4~9打鍵程度の短い単語が間断なく出題されます。そのため、極度にワード慣れしたとしても、打鍵動作の解凍(4章 2節を参照)は一度に5~10打鍵分しか行えず、何度も繰り返し行う必要があるため、組立の能力は変わらず要求されることになります(現在の私の場合、慌ただしすぎてそもそも解凍というようなレベルにたどりつけていない、という感じですが)。よって、単語単位でのランダム出題は、ワード慣れによる組立への負担低下が少ない、という特徴をもたらすと考えられます。ただし、ワード慣れしていない段階で、例えば e-typing とどちらの組立負担の割合が大きいか、等という問題はまた別の話なので、注意が必要です。

\subsection{打ち始めの数文字}

打ち始めの数文字は、他の競技と同じく、認識・組立の素早い始動が重要となります。1トライアルの打鍵数は400文字と比較的長めなので、全体の中での重要性は低めですが、対策は必要です。私の場合、最初の数文字だけは流れる文字ではなくローマ字を読んで反射的に打つようにすることで、認識・組立しか働いていない時間(動作能力が無駄になっている時間)を減らす、という方針をとっています。とは言えこの方針は、この後で、ローマ字から流れる文字への視線の移行と、打鍵動作の組み立て方の変更(アルファベットを見て反射的に打つやり方から、打鍵チャンクを組んで打つ通常のやり方への変更)が必要になり、その際、認識にある程度の負担がかかります。こういったメリット・デメリットを考慮して、自分に向いた方針を探すのが良いと思います。

\subsection{1ラップ目}

1ラップ目のうち、打ち始めの数打鍵~十数打鍵程度を除いた部分です。まだ始まったばかりということで、認識・組立・動作ともに安定しにくく、大崩れしやすい局面です。\key{Esc} を多用する(1ラップ目で失敗したら中断する)人でも、例えば20回に1回しかまともに1ラップ目を打ちこなせないのと、10回に1回はまともに打ちこなせるのでは、長期的に見れば記録の出しやすさに大きな違いが出てくるため、1ラップ目は絶対に重要です。

認識・組立・動作を早い段階からバランス良く働かせるためには、まずは速度を少し押さえるのが無難かと思います。特に認識(先読み)が暴走しがちですので、そこに気をつける必要があります。先読みの量は、初めは比較的少なめで、後から少しずつ多くしていくのが良いかと思います。また、何が何でも記録を狙いたい、1ラップ目から強引に加速したい、という場合はこの限りではありません。

\subsection{2~7ラップ目と8ラップ目前半}

2~7ラップ目と8ラップ目前半では、基本的には認識・組立・動作の全てが並行する状況が続きます。特筆すべき点は、この部分は300打鍵以上の長さがあり、その間様々に状況が変化するため、ある程度臨機応変に認識・組立・動作のあり方を変えていく必要があるということです。例えば、行が変わる直前に先読みの量を増やして視線移動の認識負担に備えることができます。逆に、行が変わる直前に動作速度を下げて認識の余裕を作る、という無難な方針をとることもできます。物理的に打ちにくい単語が来た時に、先読みを少し減らして組立の精度向上に意識を割き、丁寧に打ちこなすこともできます。もし「ギリギリ更新ペースだ、ここで少しでも減速したら更新を逃す!」という状況であれば、逆に組立を疎かにしてとにかく感覚に任せてゴリ押す、という方針も悪くありません。ちなみにこれらの方針の決定軸としては、認識をどの程度先行するか(先読みの量と速度)・組立の精度(丁寧に打つか感覚で打つか)・動作速度 の三つだけ考えておけば、とりあえず十分でしょう。

このように、現在の局面とそのトライアルの狙い(記録更新、正確性向上、速度向上、ワード慣れ、安定性向上……)を考慮に入れて、臨機応変に打ち方を変えることがタイプウェル攻略の鍵です。もちろん、こんなことをじっくりと考えながら打つことはできません。普段からトライアル毎の狙いをしっかりと決めて打ち、打ち終わった際に何か引っかかれば、リプレイを見ながら「この局面ではどうすべきだったか」ということをぼんやりとでもいいので考えましょう。それを繰り返すことで、少しずつ的確な状況判断を下せるようになっていくと思います。

\subsection{打ち終わり}

認識は組立・動作に先行している(先読みを行なっている)ため、打ち終わりの数文字~十数文字では認識作業を行う必要はなく、組立・動作に集中することができます。このため、多くの場合はいわゆるラストスパートをかけることが可能になります。自分の動作速度の限界に迫るつもりで、全てを出し切りましょう。特に、ラストスパートをかければ更新できそうな場合などは、積極的に狙っていくべきだと思います。ただし、更新が間近に迫った最終行は非常に緊張するため、付け焼刃のラストスパートは大失敗につながりやすいです。普段から意識的にラストスパートを行なって、いざという時に緊張して大失敗することのないようにしましょう。

\subsection{意識的/無意識的なペース調整}

3章 5節「減速」でも触れましたが、実際の打鍵ペースは、認識・組立・動作の能力だけではなく、意識的/無意識的なペース調整に依存しています。特にタイプウェルでは目標インジケータや経過時間表示などによりペース把握がしやすく、一定のペースに保ってしまいがちです。意識的なペース調整はいいのですが、無意識のペース調整はかなり厄介で、しばしば記録向上の妨げになります。これについては6章4節「無意識に設定している天井を外す」で触れようと思います。

\subsection{モードごとの違い}

ここまでは基本常用語を念頭に置いて書いてきましたが、その他のモードも重要です。以下に常用語を除く3モードの特徴を挙げてみます。

\subsubsection*{カタカナ語}

日本語にはあり得ない珍しい文字パターンの単語が多く、組立の負担が高くなりがちです。また、私はむしろ動作はしやすい方と思っていますが、運指の相性、ハイフン慣れなどにより、動作に大きな負担を感じる人もいるようです。ちなみに、小指を使用する標準運指よりのタイパーはカタカナを得意とする傾向があるように感じます。

\subsubsection*{漢字}

とにかく「読みにくい」というのが特徴で、認識への負担が非常に高いです。私はローマ字読みを採用していますが、ローマ字読みに熟達しても、漢字を完璧に覚えても、恐らく認識への負担は依然として高いと思います。

\subsubsection*{慣用句・ことわざ}

ワードが長かったり短かったり、助詞が含まれていたりして、認識・組立・動作全てやりにくいと思います。また、国語Rでは珍しく左手の負担が大きい(特に\key{W} が多い)ため、動作への負担も大きめです。

\section{どのような練習をするか}

タイプウェルの記録を向上するためには、基本的には練習あるのみです。しかし、どのような練習を、どのようにすべきか、という点では、色々と工夫の余地があります。この章では、タイプウェルの記録を高めるために“どのような”練習をすべきか、もしくは練習の内容をどのように決めるべきか、といったことについて、私の見解を書いていきます。

\subsection{ボトルネックを見極める}

まず第一に、自分のタイピングについて、ボトルネック(最も足を引っ張っている要素)を見極めることが大事です。これまで説明してきたように、何が足を引っ張るかは状況によって様々に変わりますが、それでも特に足を引っ張っている要素がきっとあるはずです。それを見極めることは難しいと思いますが、必ずしも正解を見つけられなかったとしても、自分にとっての課題を積極的に探していくようにするべきです。漠然と「タイピングを速くしよう」と思って練習するのではなく、「○○の能力を伸ばそう」と具体的な課題をイメージして練習した方が成長しやすく、しかも楽しいと思います。

ボトルネックを見極めるためには、ミスをしたり減速してしまった時に、「どうして詰まってしまったのか」を自問自答していくことが必要です。以下の認識・組立・動作の節で、それぞれの要素が原因で起きてしまう失敗や、改善する方法について、いくつかの例を挙げていきます。

\subsection{認識能力を高める}

認識に由来する失敗に敏感になるには、詰まって\key{Esc} した直後などに、先ほどまでの自分が「今何を打っているのか」「次に何を打てばいいのか」がちゃんと分かっていたかどうか、自問してみてください。認識が問題なく行われていればどちらも自然と把握できているはずですが、実際は、手元の打鍵に意識を割きすぎたり、先読みが先行しすぎたりして、どちらかが疎かになっていることが多いです。

タイプウェルにおいて認識能力を高めるということは、主に「先読みを適度に、かつ正確に行えるようにする」ということです。常に先読みに若干の意識を割き、今打っている内容もしっかり保持する必要があります。そのためには、常にある程度集中して、適度な量の先読みを保持するよう心がけて打つようにしてください。「常に心がける」というのは難しいことで、私もすぐ疎かにしてしまい、その度に気持ちを入れ換える羽目になります。それでも少しずつ先読みは上達していますので、地道にやれば伸びるものなのだと思います。

\subsection{組立能力を高める}

これまで組立という要素の存在を、当たり前のように扱ってきました。しかし実際はこの要素はなかなか意識されにくく、あまり重要視していない人が多いように感じます。打鍵動作はほぼ無意識に行われていて、組立コストは動作コストに比べれば無視できるほど小さい、という考えだと思います。

しかし、私は全くそうは思いません。私の運指はいわゆる標準運指で、スペル打ち分け以外の最適化は行っていませんので、打鍵動作は比較的無意識に近いはずです。それでも実際は無意識とは程遠い、もしくは、動作を意識的に制御すること(組立の精密化)により打鍵速度を大幅に向上できる、と感じています。組立の改善によって打鍵速度を向上できる例をいくつか挙げてみようと思います。

\subsubsection*{打鍵チャンクの改善}

これは組立の改善の中では最も分かりやすいものかと思います。打鍵チャンクとは、複数キーの打鍵を一つのまとまった動作として認識する場合の、一まとまりのことです(私の造語なので、一般的な用語ではありません)。これの組み方を改善することにより、打鍵速度や正確性を向上できます。

さくさく という単語を例に挙げます。\key{S}\key{A}\key{K}\key{U}\key{S}\key{A}\key{K}\key{U}を\finger{21872187}で打つとします。これは元々非常に打ちやすく、\key{S}\key{A}\key{K}\key{U}を1つのチャンクでとらえ、自然に素早く入力できます。しかしまだまだ改善の余地はあります。ある程度組立と動作に余裕があれば、\key{S}\key{A}\key{K}\key{U}\key{S}\key{A}\key{K}\key{U}を全て1つの打鍵チャンクととらえ、8打鍵で一つの動作という感覚で素早く打ち抜けます。この動作はしっかりと組み立てておかないとすぐに詰まりますが、ある程度有用です。組立能力が高まれば、このように組立の負担を増やし動作速度を向上する方針をとることが可能になり、選択の幅が広がります。

\subsubsection*{打ち分け・最適化}

組立の負担を高めて動作の負担を下げる、一般的で有用な方法です。私は最適化は行っておらず、打ち分けもまだ未熟なので、ここで特に語ることはしません。

\subsubsection*{動作の空白地帯をなくす}

よくよく気を付けてみると、指の動きとしてはまだまだ余裕があるのに動作を疎かにしてしまっている部分が沢山あるはずです。例えば\key{Space} の前後、打ちにくい単語で減速したあと、打鍵チャンクと打鍵チャンクの間、などはそうなりやすいです。集中して先の方までしっかりと組立をしていればノータイムで打てる部分でも、漫然と打てば遅くなり、動作能力を生かしきれなくなってしまいます。認識の先行(先読み)が重要なように、組立の先行も重要です。漫然と指が動くに任せず、常に短縮できる箇所を探しながら打ち進みましょう。

\subsubsection*{空間把握能力の向上}

これも組立能力のうちの重要な要素です。例えば\key{U}(\finger{7})→\key{N}(\finger{7})のようなホームポジション外からホームポジション外への同指異鍵など、空間をしっかり把握して動作を組み立てないとミスにつながりやすい打鍵パターンは、いくつもあります。正確な空間把握もできるだけ常に気を付けるようにしましょう(実際のところ、空間というよりは指との相対的な位置関係に頼っている気もしますが)。

\subsubsection*{局所速度・局所正確性のコントロール}

打鍵パターンの打ちやすさ、状況、トライアルの狙い、余裕などを考慮して、局所的に速度や正確性を上下させる場合があります。更新狙いの最終ラップで乱打をしてタイムを稼ぐような、そのたぐいの調整を常にチマチマやるということです。これを強く意識しはじめたのは最近なので、特にまだ書けることもないのですが、ある程度余裕が出てきたら、こういった小細工も有用になってくるかと思います。

\subsection{動作能力を高める}

タイピングの要は、やはり動作能力です。これを高めるために最も重要なのは、指の器用さ自体を高めることです。しかしそれについては有効な練習法が分かっていません。今回はとりあえず、打鍵パターン別に動作能力を高めていく、ということを考えます。

\subsubsection*{苦手な打鍵パターンを意識する}

リプレイの見直し、詰まった単語の反復練習などを通して、苦手な打鍵パターンを頭に染み込ませましょう。単純にその打鍵パターンの打ち方が上手くなるだけでなく、苦手であることを強く認識することによって、トライアル内での対応(減速するとか、突っ切るとか)を決めやすくなります。

\subsubsection*{得意な打鍵パターンを意識する}

得意なパターンについても同じです。打ちやすい、打っていて気持ちいい、自然と加速できる単語を沢山見つけて、トライアル内でしっかり活用できるようにしましょう。

\subsubsection*{持久力の向上}

1回や2回のトライアルではなく十数回~数十回のトライアルを繰り返して記録を出す必要上、動作能力を長時間保つために、指や腕の持久力も重要です。持久力については 7章 3節でもう少し詳しく考えます。

\subsection{無意識的な速度制限を外す}

これまで何度か触れてきましたが、染み付いたペース感覚によって無意識に打鍵速度を制限してしまっていることがあり、これを取り払うことが時に必要になります。無意識の速度制限を取り払うために有効そうな練習をいくつか紹介します。

\subsubsection*{無理やり制限を超えてみる}

とにかく無謀なぐらい速く、物凄い乱打で何トライアルか打ってみます。すぐに「意外といけるじゃん」となることもありますし、そうでなくてもペース感覚のリセットに一役買うと思います。

\subsubsection*{自分のリプレイを見ながら打つ}

自分の直前トライアルのリプレイ、もしくは最高記録のリプレイを見ながら、それより速く打ってみます。リプレイより速く打てるようでしたら、少なくとも動作能力にはまだ余裕があったということです。この練習は手軽ですし、通常の練習効果もありますので、ぜひ定期的にやってほしいです。

\subsubsection*{イメージトレーニング}

タイプウェルのトライアルを開始し、実際のキーボードは打たずに、頭の中だけでタイピングを行ないます。できるだけ速く、しかし実際の動作をしっかりとイメージしながら行うように心がけます。自信を持って「効果がある!」とは到底言えない練習方法ですが、個人的には、ペース感覚をリセットしたり、認識・組立のあらを探すのに役立つと思っています。イメージ上なのにミスをしてしまうこともよくあり、認識・組立の雑さを感じることができます。気分転換にもなりますし、こういった練習も一興かと思います。

\subsubsection*{上級者のリプレイを見る}

自分が目指しているスタイルに近いタイパー、もしくは自分より(タイプウェルのランクシステムで)2ランクほど上のタイパーなどのリプレイを見ながら、実際に打ってみたり、イメージトレーニングをしてみたり、といった練習です。今の自分より速いペース、というのが明確に実感できるため、効果はかなりあるのではないかと思っています。また、こういったリプレイを見て上級者の打鍵をイメージすることは、モチベーションの維持や成長計画の決定に良い影響を与えると思います(7章で詳しく触れます)。

\subsection{意識的なペース調整}

無意識的なペース調整と違い、意識的に自分のペースを調整することは、安定性の向上やトライアルごとの狙いの達成などのために、非常に重要だと思います。意識的にペースを調整するためには、「何のために、どの程度のペースに調整するのか」というビジョンが必要になります。更に、このビジョンを設定するためには、自分の実力をある程度正確に把握する必要があります。

\subsubsection*{ペース調整のビジョン}

ペース調整の目的と、調整するペースの程度は、例えば「大幅に記録更新するため、安定性を犠牲にしてもとにかく最高速で打ちたい」「実力的には問題なく更新できそうなので、最高速は出しきらず堅実に打ちたい」「正確性を高めたいので、ミス1\%以内を出せる速度で打ちたい」など、いくらでも考えられます。このビジョンを常に持つことは、長期的に見るとかなり重要だと思っています。前提としては、当たり前の話ではありますが、速度を上げるほど正確性・安定性は下がる傾向にある、ということに注意が必要です。

\subsubsection*{自分の実力を正確に把握する}

ペース調整の目的が定まっても、自分の実力が把握できていなければ、どのペースに調整していいか分かりません。どの程度の速度で打てば、どの程度の安定性と正確性を実現できるのか。これをしっかり把握しておくことが重要になります。もちろんある程度打ち込んでいるタイパーであれば、大まかには自分の実力を把握しているはずです。しかし、その大まかな把握を超えた正確なペース感覚は、ある程度意識して練習しなければ得られないものだと思います。目標インジケータや経過タイム表示、ラップタイム表示、文字が流れる速度、打鍵感覚など、様々な指標を活用してペースを一定に保ちながら、安定性・正確性がどのように変化するか、じっくりと観察するような練習が有用だと思います。ちなみに、正確なペース感覚を身につけることは、自分の状態(調子)を正確に把握することにもつながると思っています。

\subsection{まとめ}

この章ではタイプウェルの上達方法について書いたつもりですが、具体的な練習方法の記述が少なく、曖昧すぎて、役に立たなさそう、と感じる方も多いかもしれません。確かに曖昧で、タイプウェルの上達に直接は役に立たないと思うのですが、こういった曖昧なこと、特に「何の能力を伸ばしたいのか」ということを常に気にとめて、後はひたすらトライアルを繰り返す、というやり方が、基本的ながら理想の練習だと私は考えています。

\section{どのように練習をするか}

\subsection{“どのように”の重要性}

「継続は力なり」という言葉があります。タイピングもその通りで、長期間練習を続けなければ一定以上の上達は望めません。タイプウェル国語Rで言えば、全くのゼロから総合ZJランクに到達するまでに、とてつもない才能と努力(と自由な時間)が無い限り、最低でも1年は要するでしょう。それ以上のレベルに到達するには、数年~十数年という単位での練習が必要になります。これだけ練習期間が長くなってくると、どのように練習をするのか、つまり練習時のコンディションや、モチベーションの維持、成長計画といったことが重要になってくるはずです。この章では、そういった“どのように”の部分に焦点を当て、効率よく、また楽しく練習を続けるための方法を考えていきます。

\subsection{コンディションを整える}

「記録を出すため」に必要なのは言うまでもなく、「成長効率を高めるため」にも、コンディションの良い時に練習することが重要です。打鍵感覚は実際の日々の打鍵の積み重ねで作られていくものなので、調子が良い時に重点的に練習を行い、自分にとってできる限りの、速く・正確な打鍵を積み重ねていくことが大事だと思います。ではタイピングのコンディションとは一体どんなものでしょうか。個人的には、コンディションは脳の働きと身体の働きの二つに分けられると思います。それぞれについて、気を付けるべき点を挙げていきます。

\subsubsection*{脳の働きを高める}

脳の働きを高めるためには、十分な食事、十分な睡眠、十分な目覚め の三つが重要だと感じます。食事と食事の間、特に朝食前は血糖\footnote{血液中に含まれるブドウ糖。ブドウ糖は脳のエネルギー源。}が少なくなり、集中力が低下する\footnote{参考 日本栄養士会 Web サイト\url{http://www.dietitian.or.jp/consultation/b\_01.html}}とされていて、十分な食事が大切であることが分かります。次に睡眠についてですが、経験上、睡眠が少しでも不足していると、タイピングには明らかに悪影響が出ます。私の場合、タイピングを始めてから睡眠の不足を感じたら、(時間に余裕さえあれば)1~2時間ほど仮眠してから練習します。十分な睡眠はそのくらい大切だと思っています。睡眠からの十分な目覚めも重要です。本来は、朝起きて食事をして身体を動かして……という風に健康的に目を覚ますのが理想だと思いますが、中々そうもいかないので、私の場合はコーヒーを飲むことで無理やり目を覚ましています。糖分をとることが脳の働きに良いとも聞く(簡単に調べたところあまり信頼できなさそうですが)ので、練習を始める時にコーヒーとお菓子を用意して、少しずつ食べながら練習するようにしています。

\subsubsection*{身体の働きを高める}

これも非常に重要です。整えるべき身体コンディションとしては、体温(血行)、爪、姿勢、疲労の四つが重要かと思います。体温については、日本人のタイパーであれば誰しも「冬の寒い時期に指がかじかんで打てない」といった経験をしたことがあり、重要性はよく分かっていることと思います。部屋を暖める、風呂上がりに打つ、しっかりと食事をとるなど、体温を保つためにしっかり対策をとりましょう。爪については、個人の好みはあると思いますが、私の場合できるだけ短く切った方が上手くいきます。また、「常にほぼ同じ長さに保つ」ことが、打鍵感覚を揺るぎないものにするために重要だと思います。爪の長さが変わると打鍵感覚は大きく変わりますので、長さが変わる度に打ち方を微調整しなくてはいけなくなります。姿勢についても同様です。机のどの辺りにキーボードを置くか、キーボードの角度、椅子の高さ、画面と目の距離……。それぞれどんな状態を好むかは人それぞれですが、常にほぼ同じような姿勢で打つことが重要だと思います。疲労については、非常に重要だと思いますので、次の節に分けて詳しく考えます。

\subsection{疲労}

タイピングは常に自分の限界に近い速度で指を動かし続ける競技なので、動きは地味でも、長時間の練習を行えばかなりの疲労がたまります。手、前腕、肩、首などに特に疲労があらわれます。私の場合は、本気の練習を2~3時間ほど続けると少しずつ疲労が意識できるようになり、6~7時間で肩の筋肉に刺すような痛みが出て、手と前腕にある指を動かす筋肉の疲労により、打鍵速度・正確性が明らかに低下します。つまり、密度の高い練習を長時間行えば打鍵速度・正確性が下がり、練習の質が下がってしまいます。「質の高い練習を」「長時間」行うことが上達の近道ですが、疲労がその邪魔をする、ということになります。この節では、厄介な疲労の問題とどう付き合うかを考えます。

\subsubsection*{長時間練習の良し悪し}

疲労がたまると言っても、たまにしか練習しないのであれば(そして\ruby{腱鞘炎}{けんしょうえん}等のリスクを無視するのであれば)、筋持久力の限界まで練習すればいい話です。しかし、毎日、もしくは頻繁にタイピングの練習を行う場合、そういうわけにもいきません。日をまたぐ疲労が存在するからです。一定以上の練習を行うと、主に手と前腕の疲労が翌日以降に残ってしまい、練習の質が下がってしまいます。また、記録を狙っている場合、前日までの疲労が残っているようでは話になりません。

ならば疲労がたまらない内に練習を切り上げてしまうべきか、というと、それも疑問です。練習時間が長ければ長いほど単純な練習効果は高くなるでしょうし、私の経験からすると、長時間連続して高負荷の練習をした方が、(短期的な)練習効率は非常に高くなるからです。つまり実際には、翌日以降に練習の支障になるほどの疲労を持ち越さない程度に、しかしできるだけ長時間まとめて練習をする、というように、上手くバランスをとっていく必要があります。

さて、とりあえず「翌日以降の練習の支障にならないよう、適度な練習量におさえる」という結論に至りましたが、「筋持久力の向上」ということを考えると、話は更に複雑になってきます。練習効率(長時間にわたって質の高い練習をできるようにする)や長文競技のタイム向上のために、長時間の練習を通して筋持久力を向上させたい、という方針があり得ます。結局、長時間練習には「筋持久力向上・当日の練習効率上昇・翌日以降の練習効率下降」という三種類のメリット・デメリットがあり、これらのバランスを取りながら、自分にとって望ましい量の練習を行なっていく、ということになるでしょう。こういった点については、筋持久力の向上がどの程度であるとか、年齢による差とか、そういった運動生理学的な知見が組み合わされば、もう少し実のある結論が出てくるのかもしれません。

\subsubsection*{疲労コントロールの具体例}

個人的に理想だと思っている練習量決定の方針です。完全に実現できているわけではないですが、できるだけこれに沿う形にしようとは心がけています。ちなみに持久力の向上はほぼ無視しています。かなり迷いましたが、怪我をするリスクは非常に恐ろしいため、こういう方針となりました。

\begin{itemize}
 \item 翌日に疲労を残さない(基本的に3~4時間で切り上げる)
 \item 翌日、二日後に練習できない場合は長時間練習する
 \item 記録狙いの場合は練習量を減らして(2~3時間に)コンディション調整
 \item 怪我のリスクを減らすため、6時間以上は打ち続けない
\end{itemize}

練習のしすぎは、\ruby{腱鞘炎}{けんしょうえん}(生活習慣の乱れ、練習のしすぎなどによって引き起こされる腱や腱鞘の炎症。タイパーの難敵)にもつながります。短期的な練習効率向上のためにも、長期的に練習を続けるためにも、しっかりと疲労をコントロールするようにしましょう。

\subsection{ビジョン・成長目標・短期的課題}

効率良く上達するため、というより、タイピングをより楽しむために、ビジョン・成長目標・短期的課題の三つを設定することが大事だと思っています。これだけだと意味が分かりにくいですが、つまり、長期的な「どんなタイパーになりたいか」というビジョン、中期的な「どんな記録を達成したいか」という成長目標、そしてそのために達成しなければならない課題、という意味です。

\subsubsection*{ビジョン}

ビジョンとは、「どんなタイパーになりたいか」という、タイパーとしての志向です。正確性タイパーになりたい、乱打タイパーになりたい、タイプウェルで総合1位になりたい、e-typing で天下を獲りたい、○○さんのようなタイパーになりたい、100才までタイピングを続けたい、他のタイパーに好かれたい、などなど、もちろん何でも構いません。自分がどんなタイパーになりたいか、ぼんやりとでも明確にでも、とにかくイメージを持つことが大事だと思います。私の場合、若い頃はとにかくたにごんさんに憧れていたので、どの競技でもできるようになりたい、正確性は高めがいい、などと思っていました。今は海外のトップタイパーに憧れて英文タイピングに傾いたり、でもタイプウェル国語R ももっと上達したかったりと、いまいち方向性が定まっていません。もっとビジョンを明確にしなければいけないと考えています。

\subsubsection*{成長目標}

中期的な「どんな記録を達成したいか」という目標も大事です。具体的な記録ではなく、「どのように、どの程度上達したい」といった、能力の成長に関する目標でも構いません。ビジョンとの整合性(成長目標がビジョンの達成につながっているかどうか)は無い場合もよくあると思いますが、それでもいいと思います。自分が達成したいと思う記録や順位、上達したいと思う具体的内容をしっかりと持って、それに向かっていくことが、上達のためにもタイピングを楽しむためにも非常に重要です。

\subsubsection*{短期的課題}

成長目標の達成のために、まず達成するべき様々な短期的課題のことです。例えば現在の記録が国語R 総合得点 1125000(総合 XX )で、成長目標が総合 ZJ だったとします。そこで、5000点を伸ばすために「常用0.5秒、カタカナ2秒、漢字1秒、慣こ1秒更新」を目安にしたとします。では、これを達成するためにはどの能力をどう伸ばせばいいのでしょうか。これを現在の自分の実力と相談して考えます。例えば「常用は大分安定してきた。次はもっと加速できる場所を増やそう」とか、「今のところカタカナが全然安定していない。苦手なパターンを総ざらいして、安定性を一気に向上しよう」とか、「まだ読みで迷ってしまう漢字が十数個ある。まずはそれを完璧に覚えよう」とか、具体的な課題が見えてくるはずです。このように目標を達成するための具体的な課題をいくつも見つけて、それを一つずつ達成していくことが、着実に上達していくためのコツだと思います。

\subsection{モチベーション}

タイピングとは長い付き合いになるため、モチベーションの維持は非常に重要です。タイピングは趣味ですし、無理やり続けることはないのですが、楽しく練習を続けられるのならばそれが理想です。これこそ他人が口を出すべき問題ではないのかもしれませんが、楽しくタイピングを続けるために重要だと思っていることを、いくつか紹介してみます。

\subsubsection*{能力の細分化、伸びしろの発掘}

前節の「短期的課題」で少し触れましたが、手に入れたい能力を細分化して、様々な具体的な課題を見つけることで、「練習が報われる」機会を増やし、日々の練習を少し楽しいものにすることができます。「総合ZJを出したい」とか、「動作速度を高めたい」というような大雑把な目標は、順調に達成できている内はいいですが、成長が停滞してきたとき、不要な重圧になりがちです。何をすればその目標を達成できるのか分からなくなってきて、「総合ZJなんて一生無理なんじゃないか」とか、「もう動作速度は限界に達してるんじゃないか」とか、どんどんネガティブな方向に進んでいきます。こういう時、「動作速度」という能力を細分化していくと、何やら光が見えてきます。「\key{Space} をもう少し素早く打つ」「\key{H}\key{O}\key{U}というパターンをもう少し素早く打つ」「同キー連打をもう少し素早く打つ」などと細分化していって、こういう小さな課題を少しずつクリアしていけばいいんだ、と考えると、目標を達成できるような気がしてきます。こんなことを考えなくてもどんどん上達していける人もいるかもしれませんが、上達が停滞している時は、積極的にこういった工夫をしていくことが大事だと思います。自分の能力のちょっとした成長に敏感に気付き、それを喜べるようになると、タイピングがとても楽しくなるはずです。

\subsubsection*{他のタイパー達との交流}

ストイックに自分と戦い続けるのみ、ランキングで実力を競うのみ、というのもいいですが、タイパー同士で交流するようになると、やはりタイピングの楽しさは増すと思います。私の場合は twitter とブログ(あまり更新していませんが……)で、他のタイパーと交流を持てるようにしています。他のタイパーから様々な刺激や知見が得られてとても楽しいですし、上達にも役立っていると感じます。例えばこのタイピング合同誌へ参加させていただいたことも、 twitter をやっていたからこそ得られた機会です。こうやって他のタイパーと気軽に交流できるのはとてもありがたいことだと思います。

\section{おわりに}

以上で、今回の記事は終わりです。ここまで拙い記事を読んでくださった皆さん、本当にありがとうございました。この記事では、私が普段タイピングについて考えていることというか、私のタイピング観のようなものの体系をまとめようと試みました。しかし予想以上に自分の中で考えがまとまっておらず、また単なる思い込みのような根拠の無い部分も多く、結果的に、何が何やら分からないものになってしまったような気がします。非常に恐縮です。ともかく、長い間タイピングを(特にタイプウェルを)練習する中で少しずつ形成されてきた考えを、そのままドバッと詰め込んだような形になりました。ここまで本格的にタイピングについての文章を書いたのは初めてなので、これを機にぜひ独りよがりな考えは正し、問題の無い部分は更に深めていきたいと思っています。ブログや twitter などを通して、今回の記事について色々なご意見・ご批判をいただければ有難いです。

最後に、このタイピング合同誌を主催し、執筆作業を引っ張ってくださった tomoemon さんに、心からの感謝を表します。
