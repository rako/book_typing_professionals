\section{タイプウェル}

\subsubsection*{GANGAS 氏}
\noindent タイプウェル (TW) シリーズ開発者。
1998 年より、競技的使用に耐えるフリーのタイピングソフトを多く開発・公開。
国内一の規模のランキングを過去から今現在に至るまで運営し、タイピング界を支え続けている。

\question{はじめに、GANGAS さん個人とタイピングの馴れ初めについて教えて下さい。}

\answer{GANGAS}{パソコンのことを「マイコン」と呼んでた頃(30 年ほど前)、そのマイコンでもぐら叩きのような単純なタイピングゲームを作って遊んだのがタイピングとの付き合いの始まりです。}

\question{非常に年季が入っておられますね。美佳タイプ、そしてそのランキングサイト Typing Attack が国内競技タイピングの興りであると伺っていますが、GANGAS さんも影響を受けたのでしょうか。}

\answer{GANGAS}{タイプウェルの制作面で影響を受けました。国語Rと英単語のレベル設定の違い(タイム差)は Typing Attack を参考にしています。ただ、美佳タイプや Typing Attack の存在を知ったのは GANGAS でさえないランキングを開始した後だったと記憶しています。当時頭ひとつ抜けてトップを走っていたのは yuki さんで、その yuki さんが dqmaniac に名前を変えて GANGAS に参加されたころから GANGAS もTyping Attack 出身者でにぎわい始めた気がします。私も Typing Attack にはタイパーとして参加はしましたが、熱心ではありませんでした。}

\question{なるほど、タイプウェルとそのランキングの興りは美佳タイプや Typing Attack とは独立だったんですね。では、タイプウェル制作の動機やきっかけというのは何だったんでしょう。}

\answer{GANGAS}{ハッカーにずっと憧れていて、ハッカーになるにはプログラミングができなければ話にならないし、タイピングが速くなければかっこ悪い。じゃぁまずはタイピングソフトでも、てな感じですかね。これと言ったきっかけはありません。結局ハッカーにはなれませんでしたけど(笑)}

\question{タイピングが速いとハッカーっぽいというのは今でもありますね(笑)実は僕も競技タイピングをはじめた最初の動機はそれだったクチです。では続けて、公開初期のお話を伺いたいです。2000 年頃からタイプウェルの公開が始まったと把握しておりますが、当初はこんなことがあった、という出来事などあればお聞かせ下さい。}

\answer{GANGAS}{初期のタイプウェルは現在のオリジナルの「のみ」と「混在」だけしかない貧相なものでしたが、1998 年にベクターに登録してすぐに "PC Life 創刊号" で雑誌デビューを果たしています。日付は定かではありませんが、翌年には「のみ」と「混在」だけのランキングを開始していたと思います。}

\question{1998 年! 初期のものを含めるとさらに歴史があったのですね。}

\answer{GANGAS}{で、初期のランキングをリードしていたのがオリジナル(総合) ZH のたけひささん。確かレベル A あたりから参加されて瞬く間に頂点に上り詰め、その後は新記録樹立の連続です。当時は彼の成長を確認するのが楽しくて仕方ありませんでした。私にとっては彼が競技タイピングの先駆者です。}

\question{たけひさ氏、古参タイパーの方がお名前を挙げられているのを伺ったことはありました。当時は圧倒的な存在だったんですね。「タイプウェルオリジナル公式最速記録まとめ」\footnote{\url{http://bit.ly/u3KUcB}}にも、2001 年頃までたにごん氏とデッドヒートを繰り広げる勇姿が残っています。……では公開初期ではなく、全体を通して見ると、公開・開発を通して印象深い出来事など何かありましたでしょうか。}

\answer{GANGAS}{タイプウェルユーザーにはプログラミングの知識をお持ちの方がとても多くて、開発にあたっては方々からずいぶん助言をいただきました。出来事ではありませんが、ソフト内に不正対策を施すと、こんな抜け道があるとか、そもそも記録ファイルのフォーマットが単純なので簡単に偽記録ファイルが作れると言われてみたり、タイパーは恐ろしい(笑)と何度も思いました。}

\begin{table*}
\begin{center}
\caption{2003年のタイプウェル更新履歴}
\label{gangas:history}
\begin{tabular}{ccl}
\hline
更新日 & ソフト & 更新内容 \\
\hline
2002.05.23 & 国語R & 1.1.5β1 で「月別起動履歴」機能と詳細記録のグラフ化機能を追加 \\
2002.09.29 & 国語R & 1.1.8β1 でトップ 99 を出力する機能を追加 \\
2003.01.09 & 英単語 & 1.2.6β1 で単語帳機能を追加 \\
2003.02.01 & 英単語 & 1.2.6β3 でミス1%以内・ノーミス総合成績を表示可能に \\
2003.03.01 & 国語K & 1.1.7β1 で国語R換算機能を削除 \\
2003.08.04 & オリジナル & 1.7.1β3 で記録詳細のグラフを点から点線に変更 \\
2003.11.09 & 国語R & 1.2.9β1 で全国ランキング表示機能の追加 \\
2003.12.14 & 英単語 & 1.3.3b2・オリジナル 1.7.4b2 で「経過タイム等表示設定」機能を追加 \\
\hline
\end{tabular}
\end{center}
\end{table*}

\question{僕自身そういうことに燃えた時期もあったので……恐縮です(汗)さて長い歴史を持つだけでなく、タイプウェルは現代の国内競技タイピング文化において、今でも中心的な存在であり続けているように思います。このように中心的なソフトウェアを開発、ランキングを運営されている感想や、思いなどはありますでしょうか。}

\answer{GANGAS}{私のような未熟者が粗末なソフトを晒してこれほど大勢のユーザーを抱えても大丈夫なのかとずっと自問しています。自分では中心的とは思っていませんが、それでも運営側で競技タイピングを引っ張りたいという気持ちは多少あります。そうなるにはきっとどなた(タイパー)とも一定の距離を置きつつ、きっちり運営している姿勢を見せる必要があるんだろうとも思っています。}

\question{確かに、中立的な立場で、粛々と運営をされているイメージは強くありますね。多大な貢献という言葉では表しきれないものがあります。いつもありがとうございます。では今度は、タイプウェルの機能など、やや細かいお話に移りたいと思います。結果詳細において、グラフ表示やトップスピード、ワーストスピード、ラップというような概念がありますね。これらは競技タイピングにおける評価の指標・視点を提示したという意味で、文化的に非常に画期的だったと思うのですが、何か参考にしたプロダクトや、他の方のアドバイスなどはあったのでしょうか。}

\answer{GANGAS}{特になかったと思います。}

\question{すごすぎて頭が上がりません。結果詳細表示もそうですが、ぱっと説明しきれないほど多くの機能が今では搭載されています。やはり初期のタイプウェルは機能も少なかったのでしょうか。今では当たり前のこの機能はこの頃についた、など何かあれば伺いたいです。}

\answer{GANGAS}{わかる範囲で調べてみました(表\ref{gangas:history})。こんなところでしょうか。現在のタイプウェルは初期の頃からさほど変っていない気がします。}

\question{おお、気の利いた機能が追加されていったことがわかりますね。しかし本当にコアになる部分は最初から完成されていたとも読み取れます。ではこういったタイプウェルの機能などの要素で、うまく作れたと自信がありアピールしたいもの、逆にここはもっとうまくやれたのではないかという気になっているものなどありますか。}

\answer{GANGAS}{目標インジケーターと記録詳細画面のグラフは自分で気に入っています。苦手語句練習機能は私自身全く使っていません。大幅に改良する余地があります。}

\question{グラフとインジケーターはやはり最高ですね。似たような質問になりますが、実装されているのですがあまり注目されていない機能で、もっと活用されて欲しいものなどはありますか。}

\answer{GANGAS}{特にありません。ただ、私自身がアピールするのではなくて、タイプウェルの特徴や機能をすべて網羅して紹介していただけるサイトがあったらありがたいなぁと思っています。この機能はこう利用すると便利だとかこの機能は意味がないとか、ユーザーの評価がたくさん入った紹介をしていただけたらと。}

\question{実は今回の同人誌の別の記事内で、少し機能紹介的なものを書かせて頂いております。が、まったく網羅には至らないですし、やはりそういうコンテンツは誰でも気軽に参照できる Web ページとしてある方が便利ですね。これを読んだ現役タイパーの方で興味もたれる方もいると思いますし、この機会に模索していけたらと思います。……では少し方向を変えて、出題ワードについて。タイプウェルと言えば出題ワードに凝っているイメージがあります。例えばカタカナ語の地名などは現実の変化にあわせ頻繁に改訂されていますね。このあたり、何かこだわりがおありなのでしょうか。}

\answer{GANGAS}{もう後には引けないという意地みたいなものですかね。地名が不評なのはわかっていますから、練習したくないワードを除外する(1-10\% 程度)機能の必要性をかなり前から感じています。なかなか実行に移せませんが。}

\question{特定ワード除外機能、需要は高そうです。しかしランキングのことを考えると配慮も必要そうですね。そしてタイプウェルのワードセットは今やタイパー内である種の「教養」になっているように思います。例えばタイプウェル国語 R (TWJR) の基本常用語に収録されている単語は、最適化されている可能性が非常に高い。この資源をぜひ生かしていきたいと思うのですが、このワードセットを再利用することは問題ないのでしょうか。公式ページに明記されているのを存じ上げないので、この辺のポリシーを伺いたいです。}

\answer{GANGAS}{問題ありません。私の了承を必要とする性質のものでもないですし。}

\question{ありがとうございます。ワードといえばオリジナルにも「ワード」と呼べるような出題文字列の傾向があることが知られてきています。このような傾向は意図的に入れられている要素なのでしょうか。また、これを利用した攻略(時間帯を固定するなど)についてどう思われますか。}

\answer{GANGAS}{意図的なものではありません。オリジナルに限らずワードや文字の出題は基本的に VB\footnote{プログラミング言語のひとつ。} の rnd 関数\footnote{VB6 でランダムな数値を得るために使われる標準の関数。}の出力に依存しています。時間帯によって傾向が変化するのも rnd 関数の仕様です。傾向を利用する攻略についてはタイプウェルを知り尽くした人の特権として「あり」だと思っています。出題にかかわるコードを長い間いじってないのもその辺を考慮してのことです。もちろん固定文字列の生成まで辿り着いてそれを利用してしまったらそれは言うまでもなくアウトです。}

\question{あまりにもハッキリ傾向が出るので、てっきり意図的なものと思っていました。フェアに「攻略」と言える範囲であれば問題ないということですね。ワードに関係したところで、個人的に最近もっとも嬉しかった機能追加に Detail.txt の出力があります。このようなインタフェースが整備されることで、タイプウェルを中心に他開発者達で新たな文化を築いていける可能性が開けてきますね。「管理人メモ」にもありましたが、今後はこうした外部連携を期待していくということなのでしょうか?}

\answer{GANGAS}{そうですね。タイプウェル周辺ツールが増えることを期待していますし、楽しみにしています。}

\question{過去にも「練習実績 Plus」などタイプウェルの出力ファイルを利用してタイプウェルに似せた画面で機能を拡張したものなどがありました。最近ですとタイプウェル風の画面でタイプウェルのワードを打つ打鍵トレーナなどもあります。GANGAS さんとしてこのようなソフト、「周辺ツール」「二次創作」文化についてのポリシーはどうお考えでしょうか。}

\answer{GANGAS}{GANGAS を盛り上げていただけるものは大歓迎、その逆は困るといった程度にしか考えていません。もちろん不正利用できるようなものは困ります。}

\question{ありがとうございます。僕も開発者の端くれなので、常識に則って行動していきたいと思います。では周辺ツールではなくて、今後のタイプウェル本体の機能追加についてはどうでしょうか。大きな変更を入れるような要望を出してもご迷惑かなと、個人的には思ってしまっているのですが。}

\answer{GANGAS}{機能追加や改良は要望に応じてと言うよりほとんど気まぐれでやって来ましたし、今後もそれは変らないと思います。}

\question{なるほど、軸がぶれないという意味で、とても心強いスタンスだと思います。また少し話題を変えまして、タイプウェルと言えば外せないのが公式全国ランキングです。この方向で少々伺いたいと思います。まずこれだけの間ランキングを更新し続けているというのはもの凄いことだと感じているのですが、やはり使命感と言いますか、思うところがあるのでしょうか。}

\answer{GANGAS}{ランキング運営は楽しいからここまで来れたのだと思います。使命感のようなものは特に感じていません。ランキングに参加していただいた皆さんには感謝しています。}

\question{昔は週毎ではなく毎日更新だった時期もあったと伝え聞きました。これはさすがにもの凄すぎるモチベーションですよね。}

\answer{GANGAS}{当時はさながら寝る間も惜しんでゲームに夢中になっている子供のようでした。モチベーションという言葉とは無縁の世界に住んでいましたから全然凄くありません。きっとタイプウェルは面白いとかはまるとか言われてタイパーの皆さんにおだてられてたんでしょうね(笑)}

\question{当時から「面白」ベースだったのですね。一方、タイプウェルのランキングは審査付き、重複登録の原則禁止などこの手のネットランキングとしては非常に厳格に運営されています。このような姿勢はどこから来ているのでしょうか。由来などありましたら伺いたいです。}

\answer{GANGAS}{こうしたほうが面白い、こうしないとつまらないと考えた結果が今のシステムです。}

\question{やはり「面白」(笑)でも事実、この運営方針によってランキングは今でも大変「面白い」ですね。そういえば私個人の感覚ですが、2006-2007 年頃にトップレベルの人たちの記録更新があまりなくなり、やや下火になってしまったのか、などと危惧していたのですが、最近では新規参加者も、記録更新者もまた盛り返してきたような気がします。GANGAS さんとしてはどのような感覚でしょうか。}

\answer{GANGAS}{下火になった時は引き際を考えましたが、今はまだまだ行けるという感覚です。記録は記録を誘いますから、タイパーの皆さんには満足できる成績が出たら送信ではなく、わずかでも記録が伸びたら送っていただきたいです。}

\question{サブマリン\footnote{ランキングに掲載されている自分の記録よりも良い記録があるのに、記録を送信せずにおくこと。}は良くないと(笑)いち競技者として個人的にも、最新の記録を送っておくべしと感じますね。手元にもう少しいい記録がある、というのは慢心につながるというか、競争に対する意識を濁らせてしまうように思います。それでは、最近個人的に注目しているタイパー、記録などはありますか。}

\answer{GANGAS}{現役タイパーは皆注目しています。記録面ではやはり人はどこまで速くなれるのかということに興味は尽きないです。}

\question{親心のようなものすら感じる、ありがたいお言葉。一方、と言ってしまうと失礼にあたりそうですが、過去には勃起教教祖氏、最近ですとしろこ氏、Sean Wrona 氏など、公式ランキングには載っていないが偉大な記録を持っているという方がおられます。こうした公式に載っていない「裏」記録についてはどう思われますか。}

\answer{GANGAS}{運営側としてはランキングに参加して欲しいところですが、裏は表を引き立たせる役割を果たしますし、参加しなくても「裏」記録として公表していただけるのはありがたいです。}

\question{なるほど。現在は真のトップ記録はちゃんと公式に登録されている状態ですしね。テル氏、きる氏などトップランカーの活躍で、最高記録は今でも更新が続いています。GANGAS さんはこのような更新をどのようなお気持ちで見ておられますか。}

\answer{GANGAS}{一ギャラリーとして歓喜ですね。未知の領域を開拓して行く人は本当にすごいと思います。脳医学の研究対象として推したいくらいです。}

\question{一般人にはたかがタイピングと言われるかもしれませんが、トップ層のレベルは他のスポーツ同様に「競技」を堂々と名乗っていいレベルに達していると思います。脳と身体を同期的に動かすという観点ではよく単純化されていますし、本当に研究対象になりえそうですね。トップレベルといえば、昔はランクが XA までしかなかった、等と伝え聞きました。当時としては XA どころではない記録がどんどん出てくるとは想像していなかったということでしょうか。当時のお話など何か伺えればと思います。}

\answer{GANGAS}{全く想像していませんでした。現役タイパーでご存知の方はいらっしゃらないでしょうが、最初のタイプウェルは XA どころか SS までしかありませんでした。タイプウェルの開発を始めた時にスピードの基準を私自身にしてしまったことがそもそもの誤りでした。}

\question{SS までしかなかったというのは全く初耳でした。関連しますが、主にオリジナル用の M1-M9 の制定に続き、最近になって R1- というランクも新設されました。M はメシエ天体\footnote{M31 はアンドロメダ銀河、等というあの M。}から取ったと伺っていますが、R は一体なんなのでしょう。}

\answer{GANGAS}{天才バカボンに出てくる「レレレのおじさん」のセリフ「レーレレーのレー」から R を取りました。何となくわかりますかね? 呆れるほどすごいと言う意味を込めています。ちなみに R の上もすでに導入済みです。}

\question{予想の斜め上を(笑)バカボンは知ってますし、ニュアンスはよくわかります。そして R の上……! 心が躍ると同時に気が遠くなりますね。自らの手で「すべてのキー」のトップスピードあたりでお目にかかりたいものです。しかしそのように M に加えて R が必要になるなど、オリジナルについてはやはり、これほど記録が伸びるとは想定されていなかったということでしょうか。}

\answer{GANGAS}{そうですね。全く想定していませんでした。昔はたにごんさんの(「すべてのキー」モードにおける) ZB が驚異でしたから、それをはるかに凌ぐ記録が生まれるとは誰も想像してなかったと思います。今は、才能があればどのモードもキーをデタラメに叩くスピードに限りなく近づける、つまりレベル R 台も十分あり得ると思っています。}

\question{GANGAS さんにそう仰って頂けると一部のオリジナルガチ勢が大変やる気をだすと思います(笑)しかし一方で、殿堂ランキングについて考えると、オリジナルを究極的に極めると他の種目が不要になってしまうなど、問題点も指摘されています。こういう点についてはどのように認識されていますか。過去に一度(?)基準の改定があったように思いますが、さらに改定する予定などはあるのでしょうか。}

\answer{GANGAS}{するかどうかは別として、殿堂ランキングは改定したいとは思っています。}

\question{ランキングとは別にトップページに掲載される What's New や Pick Up を楽しみにしている競技者も多いです。あの掲載基準はどのようなものなんでしょう。}

\answer{GANGAS}{掲載基準は閲覧者によって履歴から導き出していただくのが理想です。不親切だと思われるかもしれませんが、全部を説明せずに基準とか機能とかこうなっていると誰かに「発見される」のが快感なんです(笑)}

\question{「面白い」と思います。さすがに「発見」しようがない部分として、ランキングの生成・更新は半自動・半手動であると伺っています。これは完全に個人的興味なのですが、どのように処理されているのか気になります。}

\answer{GANGAS}{参加者全員の記録はエクセルで管理しています。まずメールを一件一件確認しながら自作ツールで記録の認証と加工処理をしてそれをエクセルに反映させます。その後マクロと別の自作ツールを併用してランキング用 html ファイルなどを出力しています。メール上の記録のコピーと加工後のエクセルへのペーストは手動です。HN の変更があると前回のランキングとの比較ができなくなる分、手作業が増えたりします。GANGAS トップページの書き換えも半分が手動です。作業は全体で 2-3 時間かけてます。その気になれば全自動にして 0 時少し回ったところで更新というのも可能ですが、「○○さん××年ぶりに復活!」とか「新鋭タイパー現る!」とかつぶやきながらの手作業も楽しいものです。}

\question{今でも楽しんでいらっしゃる様子がありありと伝わってきました(笑)そのように現在も脈々と続いている GANGAS さん個人によるランキングの更新ですが、現実的な話、これを永遠に続けることはできないかとお察しします。国内にはタイプウェルランキング以上にハイレベルで豪華なランキングは存在しないので、失われるのは想像したくもありません。考えたくないことですが、個人の都合などで、更新継続が困難になった場合、何かこうした処置を取る、といったお考えはおありでしょうか。}

\answer{GANGAS}{4 週連続で記録メール数が一桁に落ちた時に更新を止めることだけは決めています。GANGAS の役割は終わったと。不測の事態で終了することもあるでしょう。いずれにせよ更新を止める時は事前にアナウンスしますが、その後のことは今はケサラ\footnote{Que sera。「なるようになるさ」の意。}です。このスタイルを引き継げる人はいないでしょうから、新たなランキングでスタートできる方を募集することになるんでしょうか。わかりません。}

\question{やはり難しいですよね。とりあえず不測の事態が起こらず、現在の状況が続くうちは問題なく継続されると伺えただけで一安心しました。さて、ではそろそろ締めということで、既に 10 年以上という長い時間タイピング界を眺めてこられた GANGAS さんから見て、今のタイピング界はどう映りますでしょうか。またタイピング界のこれから先、未来については、どうでしょう。何か思われることなどはありますか。}

\answer{GANGAS}{タイピング界というのは匿名性を保ちつつ各々主戦場を選んで孤独に戦うタイパー達の集合体なんだろうと思います。だからスポーツ競技のように苦しいけど誰かと支え合って頑張るみたいな発想はこの世界で生まれることはなく、主要な戦場がひとつ消えるだけで急速にしぼんでしまう壊れやすい存在という気がしています。大きな時間的な流れで見ると、こぢんまりとまとまっていた集団が母体を大きくしながら少しずつ霧散の方向に進んでいる印象があります。未来については、そうですね、タイパーの関心は徐々に国内から世界へ、つまり日本語入力から英語入力へシフトして行くんじゃないでしょうか。いえ、これは私の願望です。}

\question{最後に、何かご自由にメッセージを頂けたらと思います。}

\answer{GANGAS}{このような機会を与えていただいたことをとても光栄に思います。タイピング同人誌の成功を祈っています。}

\question{ありがとうございました。}
