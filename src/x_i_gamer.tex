\section{Gamer - ゲームとしてのタイピング}

\subsubsection*{あきうめ 氏}
\noindent 小学生当時から TOD において常人離れした正確性・速度を見せつけ、全タイピング界を震撼させた。「絶対神」「アンドロイド」などの異名を持つ。TOD エキスパートにして、全国記録保持者。
\subsubsection*{父・信仁 氏}
\noindent あきうめ氏の父であり、ゲーマー。鋭い考察力と的確な指導方法で、小学生当時からあきうめ氏を全面的にサポート、他に類を見ない超一流正確性タイパーに育てあげた。

\question{あきうめさんは小学生時代のイメージが強いので、僕より背が高いことに驚きを隠せません(笑)お会いできて光栄です。よろしくお願いします。}

\answer{あきうめ}{よろしくお願いします。}

\answer{父・信仁}{いつの間にかぐんぐん伸びてな(笑)}

\question{時系列順にお話を伺おうと思います。はじめに、タイピングをすることになったきっかけは、ゲーマーのつながりで DC 版 TOD を一式入手されたからだと伺っています。}

\answer{父・信仁}{ゲームのサミットで毎回来てくれている方が、使わないからと丸ごとくれまして(笑)初めは俺がやってたんです。あの頃は俺もタッチタイプができなかったから、ちょうどいいやと。それを横で(あきうめが)見ていて……興味を持ったんだと思います。ちょうどその時期に小学校の夏休みがあって、毎年実家の方で一ヶ月過ごすようにしていたので、「これ持っていって頑張れや」と持たせて、帰ってきたら……「おい何だこれ!?」という状態に(笑)}

\question{最初は、どうやって練習されたんですか。}

\answer{あきうめ}{ジェームズ\footnote{The Typing of the Dead (TOD) の主人公。}が教えてくれるチュートリアルモードで(笑)その後に、アーケードモードの練習用ステージをやってました。ドリルモードの存在に気づいたのはけっこう後なんで、最初の頃はずっと本編ばっかりですね。}

\question{本当に TOD で一から入門されたんですね。}

\answer{あきうめ}{本当にそうです。}

\answer{父・信仁}{当時もうゲーマーだったので、これだけ打てるんなら行けるんじゃないかと、ゲーセンへ。京都ジョイポリス\footnote{あきうめ氏がホームグラウンドにしていたゲームセンター。}ではくびきりゾイルさんという方がランキングを独占していて。彼の記録を駆逐するところからのスタートでしたね。初めは全然届かなかったけど。}

\answer{あきうめ}{8000 すら届かんかった、というか、ワンコインクリアすらまず出来てなかったから。}

\question{すぐにスコアアタック的なことを始めたんですね。その後わずか 2 年程度で Easy 台でカンスト 9999 点、Normal 台でも日本記録の 9858点などの記録を叩き出します。}

\answer{父・信仁}{TOD で入門して、初期にもう「ミスしちゃダメ」というのが前提の世界に入りましたから。それには KeNo さんの影響がすごくあった。}

\question{正確性重視で知られ TOD エキスパートの一人である KeNo さんですね。}

\answer{父・信仁}{(2001 年 3 月の秋葉原パセラ)オフに行ったときに、実際見て、KeNo さんにボコボコにされて、こんな化け物じみた世界があるのかと。実は当時、やる前は勝てると思ってたんです。対戦だったら、初速もこっちが速いし……とか思い上がってて。}

\answer{あきうめ}{全然 KeNo さんの方が強かったです。}

\answer{父・信仁}{あのオフの時にはカラオケ屋(パセラ)にドリキャスをみんなで持ち寄ってね。}

\answer{あきうめ}{個室 4 部屋でトーナメントをやって、勝ち上がった人は大部屋の方で決勝を。}

\answer{父・信仁}{すごかったですよ。手作りのイベントだというのに本格的で。集まったメンバーも今にして思うとそうそうたるメンバー。で、予選は勝って、準決勝は「あきうめ vs KeNo」と「dqmaniac vs Jin」。KeNo さんと当たった時は「ラッキー」と思ったんですよ(笑)これなら行けると。}

\answer{あきうめ}{「正確性だけの人なんちゃうの?」みたいな油断はあったなぁ(笑)でもやってみたら「やばっ!」って。一章で大差をつけて、五章で逃げ切ればいいやろと思ってたんですけど、一章の時点で全然敵ってなくて「ああ無理や」と。}

\answer{父・信仁}{そして五章のマジシャンで例のアレ\footnote{相手がワード打ち切りを放棄した際に自分も打つのをやめ、対戦相手にダメージを与えることを狙う技。}を使われてな。もうコテンパンにやられて、あの後は内心「KeNo 憎し!」みたいな勢いで、絶対に彼の記録(当時の TOD 日本最高記録)だけは超えてやると誓ったんです。正確性重視で上を目指そうという発想の原点はそこですね。}

\answer{あきうめ}{KeNo さんの正確性には驚いた。当時はまさかフルコン\footnote{フルコンボ。TODではミスタイプをしないで敵を倒していくとコンボがつながり、一度でもミスタイプするとリセットされる。}できるって思ってへんかったもん。1章5章6章の3ステージを全部(ノーミスで、コンボを)つなげれるっていうのは、まず誰も想像してなかったと思う。}

\question{その後は、正確性重視のスタイルを徹底するようになったわけですね。}

\answer{あきうめ}{一番初めは正確性とかそんなに見んかったけど、それからやな。}

\answer{父・信仁}{入っていく時には正確性はそれなりでいいと思うんです。ただ TOD スコアアタックとなると、そういう甘えは一切通用しない世界です。逆に(ワードの打ち切り評価)A なんて簡単に出る。初代 TOD は S がないので。だから速さなんてダメージを受けないだけあれば十分で、ひたすら正確に打つことさえできれば、あとは(スコアルール・四連スカシ\footnote{エンペラー戦でスコアを上乗せするための高等テクニック。}・被ダメによる難易度下げのような)攻略情報さえ知っていれば、誰でもあのスコアは出せるはずなんです。でもそれが出来ない。となると、正確性を意識して打つというのがいかに難しいかっていう話なんです。}

\answer{あきうめ}{Easy 台まではその理屈が通用するな。Normal 台は速度もかなりセカセカやった気がする(笑)部分部分で難所があるんで……そこさえ乗り切れば、あとは問題ないけど。}

\question{難所というか、六章 2 つめのミッションでは何体でも倒せるので、あそこで速度が必要なのかと思っていました。}

\answer{あきうめ}{ああいう場所は大事ですよね。よく覚えてないんですけど、確か 13 体か 14 体くらい倒してました。どちらかというと、ミッション自体より鎮静剤\footnote{ミッションの報酬として出るアイテム。鎮静剤は点数にならないのでハズレである。}ばっかり出る、なんて部分に嘆いてたな。ダイヤモンド\footnote{点数になるアイテム。}来ねぇーみたいな(笑)ダイヤモンドさえ出てれば Normal 台カンストやったやろ、っていうパターンも実は何回かあった。}

\answer{父・信仁}{その辺は……ゲームだからな。あとは運だけという。}

\answer{あきうめ}{回数増やすしかない。5章6章を(一度もミスタイプせず)フルコンで行けた回数なんてもう、50 回とか……いやもっとやろな。六章やった回数だけで言ったら 10000 回とかですよ。}

\answer{父・信仁}{相当のやり込み甲斐があったよな。}

\answer{あきうめ}{四連スカシも相当練習して見極めて、最後の頃はほとんどわかるようになってましたね。単に画面の外に消えるというだけではダメなパターンもあって、出て行く角度とか、球を出した瞬間のエンペラーの立ち位置とか、色々観察して、感覚的に身につけてました。五章でコンボ数を 89 に持っていく方法\footnote{マジシャン三連弾で喰らう・道中の放置すると斧をダブルで投げてくるゾンビの利用。このように五章でコンボ数を 89 にすると、ゲージが満タン寸前になり、六章のはじめの一体でライフアップによる加点を得ることができ無駄がない}も、プレイするうちに発見したり。}

\answer{父・信仁}{どういう状況でどういうアイテムが出たかのメモを毎回取ったりとか……散々やりましたよ。そこはゲーマーですから。ゲームとしての攻略的な要素もこれほどある TOD は、本当に奇跡的なゲームです。}

\question{やり込みでのスコアアタックはもちろん、大会のような一発勝負でも結果を出されていますよね。}

\answer{父・信仁}{TOD と言っても、本当に黎明期にゲーセンでやっていた人から、PC 版の TOD2004 とかを後からやった人までいるでしょう。今の現役タイパーの人は、TOD があまりゲーセンで稼働していないので、気の毒だなと思います。あの頃やってた人は、ゲーセンで、いわゆる野試合をたくさんこなしてるわけですから。たにごんさん、dq さん、Pocari さん……彼らが毎パソのような一発勝負でも力を発揮できるのは、一つにはそういう環境で鍛えて来たということがあると思うんです。今のネット対戦で慣らしてるタイパーだって、確かに速くて上手いんでしょうが、夏も冬も空調の効いた快適な自分の部屋で、慣れたキーボードで、つまり最高の環境でやっているわけです。言ってしまえば母親の胎内と同じようなものですよ。外に出て、本当に相手と面と向かって対峙して打ったりとか、そういうリアルの空気・緊張感を知らない。あの空気の中で揉まれて来た人はやっぱり皆強いんです。}

\question{確かにメンタル面や対戦での駆け引きのような要素では、古参と呼ばれる方々に全く敵う気がしません。}

\answer{父・信仁}{俺らも当時はジョイポリスだけでなくて、京都のほとんどのゲーセンに行き尽くしましたよ。大阪も行った。ひとつひとつの店で環境が違うわけです。冷房直撃の店があったりとか。}

\answer{あきうめ}{冬で寒いのに、入り口のすぐ側に置いてあったりとか。}

\answer{父・信仁}{他にも隣に音ゲーがガンガン置いてあってうるさいとか、キーボードがタバコで溶けてて打ちにくいとか……。そういう色々な所を回って、揉まれて来たんです。TOD2003 の発売記念の大会\footnote{TOD2003 秋葉一決定戦。}も、大晦日近くで寒いところに雨が降って。}

\answer{あきうめ}{(そんな気候の中)ほとんど外に出た場所で打つという最悪の環境でした。}

\answer{父・信仁}{そこでも、予選の全力疾走一発目で 36 秒が出せた。あれなんかはもう、たまたま出たわけじゃなくて、そういう経験の積み重ねの結果なんですよ。}

\question{経験が桁違いだということを痛感します。TOD スコアアタックの後には e-typing で活躍されていたと思います。e-typing についても伺えますか。}

\answer{父・信仁}{エタイを始めた頃は o-ck さんがライバルのポジションにいましたね。}

\answer{あきうめ}{彼は強かったです。あの人がいなかったら、あんなにスコア伸びていなかったと思う。光帝院(神無)も本当に……強かったなぁ。抜いても、一瞬で、次の日には抜かれているという。}

\answer{父・信仁}{光帝院な(笑)}

\answer{あきうめ}{結局何者かもわからなかった。}

\answer{父・信仁}{海外に行くから、とお別れのようなメールなどは来たんです。あのメールの送り主が本物だったのか偽物だったのかすら、わかりませんが。実力面では、あの頃は不正だなんだと言われましたが、今から見るとそうでもないし。}

\answer{あきうめ}{慣用句で 750pt くらいやった。当時はすごかったですけど。}

\answer{父・信仁}{慣用句が伸びた理由のひとつは光帝院さんの影響だよな。極端な話、自分としては、あれが実力であろうとなかろうとどうでもよかった。(あきうめ氏に刺激を与える)材料にできるから。この人は、ライバルがいるとすごい燃えるタイプなので。しかし今は攻略も進んで、レベルも上がってるんじゃないですか。}

\answer{あきうめ}{始めた頃はみんな 600 台やったもんなぁ。650 超えたらすごかった。}

\answer{父・信仁}{今は 700 後半なんてトップの人は毎週出してるようなレベルになってきてますよね。}

\answer{あきうめ}{でも今でも 800pt 以上がバンバン出てはいないんやな。毎週 850pt とかで争うレベルに行ってても良さそうなもんやけど。700 後半って自分が中学くらいのトップのスコアじゃないですか。それが今まで続いているんやなぁ。}

\question{そうですね、本当にトップレベルの人が 800pt に少し足を踏み込んだのみで、当たり前のように 800pt を出したりはしていないです。}

\answer{あきうめ}{600 台から 700 台に上がったのが俺らの世代で、そこからは止まっちゃってる感じですね。そりゃどんどん伸びにくくはなると思うんですけど、もっと行けそうなもんやけどな。}

\question{まだそのレベルでの競争というのはまだ行われていないので……。800 を出せたら間違いなくその週はトップで、神の仲間入り、という感じですから。}

\answer{父・信仁}{もう一回復帰して降臨したら?}

\question{あきうめさんの記録を目標に頑張っていた e-typing ガチ勢は、抜き返してくれなくて寂しがってますね。実は俺さんのインタビューでも、露骨に煽ってる内容が載ってるんですけど(笑)}

\answer{父・信仁}{いや、彼の執念はすごいですよね。自分も色々読んだけど……リポビタン効果で、一点集中でね(笑)評価しますよ。}

\answer{あきうめ}{記録が抜かれたのいつだったっけ。大分後になってからやったよな。}

\answer{父・信仁}{でも実は、元気ワードでは他にもそれっぽい記録も出してるんです。腕試しの元気ワードではなくて、練習モードのような場所の元気ワードなんですが…… Error 判定\footnote{旧仕様では 800 以上で Error となった。}が出て。}

\question{詳細不明の 800pt 台があったんですね。}

\answer{あきうめ}{あれ何点やったんやろ。確かにあったな。まあそんなこと言ったら……今でもやったら出せると思うんやけどな、800pt は。}

\question{おお!}

\answer{あきうめ}{俺さんか誰かが慣用句で 800pt 出したらやりますよ(笑)}

\question{煽り返してしまった(笑)}

\answer{あきうめ}{やりましょうかね、ホント。今やったらどれくらい出せるんやろってのは気になりますから。慣用句もちょっとワードが変わったそうですし。}

\answer{父・信仁}{今でも期待してくれてる人がいるんだし、また一回やってみたらいいよ。}

\answer{あきうめ}{じゃ、ある日突然ランキングに載せるかもしれないです。慣用句 800pt 超えとかで。}

\answer{父・信仁}{当時の慣用句の最高記録も、1 ミスがなかったら 800pt 行ってたし。あれは失敗だったな。}

\answer{あきうめ}{(良い記録が)出るときの感覚っていうのがあるんですよね。あの 800pt を逃した瞬間、「あーもうしばらくは無理やな」ってわかりましたし。慣用句で WPM が 800 を超えたのはあの一回だけなんです。}

\question{その一回でミスをしてしまったというのは、やはり痛恨のミスだったと。}

\answer{あきうめ}{痛かったですね。慣用句はワードが短いのでミスのペナルティが大きくて。あれがなかったら 810pt とかでしたから。ノーミスで WPM 790 台、795pt とかは何度も出してるんですけどね。通してノーミスで打てたら大抵 780pt くらい。そこにたまに神がかってうまく打ててる時というのがあって。}

\question{これは WPM で 800 くらい行ってる感覚だなと。}

\answer{あきうめ}{そう、何か感覚的にわかるんですよね。}

\answer{父・信仁}{この人スコアが表示される前、「オツカレサマデシタ」となっている間に、これは何点って言うんだよ。すると、本当に大体それに合ったスコアが出るんです。±5pt くらいの精度で。}

\answer{あきうめ}{ほんと感覚ですね。}

\answer{父・信仁}{それくらいやりまくってるんです。出たワードと自分の打った感覚でもう大体わかるっていう。}

\question{半端じゃないですね。}

\answer{父・信仁}{慣用句は今の人だと、どれくらい出せるんですか。}

\question{慣用句は他のワードと比べるとガクッと下がります。700pt 台もほぼ出なくて、750pt 以上をポンポン出せる人となると、あきうめさんの他にはほぼいないと言ってしまっていいんじゃないでしょうか。}

\answer{あきうめ}{自分は元気でゴリ押しをやるまでは、慣用句のスコアが一番高かった。文章短くて打ち切りやすいから、慣用句はけっこういいと思うんやけどな。元気ワードであれだけスコア出るなら、慣用句も行けるだろって思います。……でも(連勝記録を作るなど e-typing に積極的に取り組んでいた)当時も、確かに慣用句は一番勝つ自信があったな。慣用句の週はたとえ o-ck さんが出てこようと負けんっていう。780pt くらい出しておけば大丈夫やろ、と。}

\answer{父・信仁}{普通に出てたね。当たり前のように、練習一発目で 780pt とか。}

\question{初速を重点的に鍛えている人は、今あまりいないんですよね。}

\answer{あきうめ}{重点的に鍛えて……たんかな? まあ勝手にそうなってたんか。}

\answer{父・信仁}{そりゃ、TOD だけやってたからな。}

\answer{あきうめ}{全力疾走は本当に初速ゲーなんで、練習になりますよね。あんなのばっかりやってたからか(笑)}

\question{e-typing に関連して、少しスキル的なお話も伺いたいです。}

\answer{父・信仁}{運指・最適化に関しては全然やらせないままで。「か」「く」「こ」は \key{C} で打つみたいなことも、結局しなかった。}

\question{標準運指で、打ち分けもしないスタイルですよね。}

\answer{あきうめ}{ジェームズのせいやろ(笑)ジェームズが(トレーニングモードで)指の動かし方を(標準運指で)教えたから(笑)あれで習っていなかったら……ホームポジションに手を置くような打ち方に最終的になっていったとは思いますけど、なんか(最適化のようなことも)やってたかもしれません。(標準運指を外れるのは)\key{-}を中指で打つやつだけですからね。あれは単に(小学生当時に手が小さくて)届かなかったからなんですけど。当時は手や腕全体で動かして打ってましたからね。今やったら小指になると思います。}

\answer{父・信仁}{本当にそれだけだよな。最適化は(標準運指の打鍵スタイルが)完成されてからやろうと思うと、膨大な時間と労力が必要になりそうで……迷った時期もあったんだけどな。}

\answer{あきうめ}{理論で言ったら、親指とかも使えた方がそれは速いと思う。}

\answer{父・信仁}{「うんぬん」とかは連打ゲーだもんな。}

\answer{あきうめ}{「うんぬん」はダメですね。全部人差し指で打つんで(笑)中指で U を打てた方が速いと思いますけど……ジェームズのせいで(笑)}

\answer{父・信仁}{速さだけを追求するなら、最適化はしていかないといけないのかな。}

\question{現役タイパーでタイプウェルで席巻しているテルさんは標準運指なんですけどね。正確性も非常に高くて。}

\answer{あきうめ}{なんか通ずるものが(笑)タイプ似てるんですね。それはぜひ会ってみたいなぁ。}

\answer{父・信仁}{じゃあ標準運指にも希望はあるんだ。}

\answer{あきうめ}{標準運指の人には頑張って欲しいわ。}

\question{打ち分けはされてますけどね。}

\answer{あきうめ}{とっさに打ち方を変えるってのは難しく感じるんですよ。自分は元々のワードの表記そのままに打つんで。アルファベットを見て、そのまま。}

\question{いわゆる「ローマ字読み」ですね。では逆に日本語の文しか表示されないようなゲームだとつらいんでしょうか。}

\answer{あきうめ}{遅いですね。すごく遅いです(笑)}

\question{一時期少しだけタイプウェルでトレーニングをされていた時には日本語を読んでいたと伺っていますが。}

\answer{あきうめ}{タイプウェルはアルファベットの表示が小文字じゃないですか。あれが無理だったんですよ(笑)TOD と e-typing、あと打トレは全部大文字だったので。}

\question{なるほど、大文字と小文字の差でしたか。}

\answer{あきうめ}{話は小学生の頃に戻るんですけど、キーボードの印字って大文字じゃないですか。あの頃は画面の文字と同じ(形が印字されている)キーを押せばいいんだろうと、そうやってたんです。}

\answer{父・信仁}{本当にそこからだったよな。ローマ字とかわかってなくて。}

\answer{あきうめ}{なので、小文字が表示されても、そんな文字が書いてあるキーないぞ? と(笑)}

\answer{一同}{(笑)}

\question{TOD・e-typing と伺ってきましたが、これらの記録はどれも偉大なものとしてタイピング界の中で語り継がれています。}

\answer{あきうめ}{俺はあんまりよく知らんからな、タイピング界のことって。黙々とゲームやってたって感じですからね。そういうのはこっち(父・信仁さん)にお任せしてた感じで。それを聞いてただけです、特に小学校の頃は。}

\answer{父・信仁}{ここまでタイピング界のことを知らないタイパーも珍しいですよ。タイパーって、相手のこと色々知りたくなって、ブログとかチェックしたりとかしますよね。この人は本当にそういうの一切やらなかった。興味がないんですよね。}

\question{情報収集は信仁さんが行われていたということですね。}

\answer{父・信仁}{芸能界でいうと、タレントとマネージャーの関係みたいな感じですよ。私は宣伝とか色々やって、こっち(あきうめ氏)は黙々と(タイピングを)やる人。二人三脚みたいなもんですよ。}

\question{面白い関係ですよね。競技者と指導者の二人三脚なんて、他にそういう方は見ませんし。}

\answer{父・信仁}{始めた時点では小学生でしたしね。そういう意味で、異色といえば異色でしょう。「才能と努力どちらが大事か」と問われたことがあるんですけど、これは問いからして間違ってるんです。これらは比べるようなものじゃない。努力はして当たり前。出来るだけ全力で努力するのは当然ですよね。だから俺は、努力の代わりに「環境」で答えるんです。}

\question{環境、ですか。}

\answer{父・信仁}{たとえ才能があって、努力も十分にしたとしても、小学生が一人で打っていたら、絶対にこうはならなかったんです。子供だけでは無理ですよ。タイピングはまだ体系化されていない世界ですよね。やっているレベルはそれなりに高いんだけど、みんなそれぞれ試行錯誤している。だから指導が必要でした。場慣れについてもそう。俺がゲーセンに連れて行って、ガンガン金を注ぎ込んで。「金は出すからお前はしっかり打て」とか、小学生にそんなこと、普通しないじゃないですか。そういう意味で「環境」という要素を強調したいです。「環境が人をつくる」んです。(あきうめ氏の)「才能」がどうかは知りませんけど、俺がやったのと同じ事をやったら、「環境」を整えたら、同じようなレベルの子は絶対出てくるんです。}

\question{言われてみると、その通りだと思います。}

\answer{父・信仁}{もちろん親ばか・ばか親だということは自覚していますよ。ただ、そういう助力がなければこういう結果は絶対残せなかった。当時あきうめより一つ下で、かなり速い小学生がいて、TOD でも対戦したことがあったんです。対戦で「もっと本気出して下さい!」とか煽って来たりして(笑)いい根性をしてました。でもやはり、環境が変わるとやめてしまう人が多くて……もったいないと思いますね。}

\question{あきうめさんの競技スタイルは、ゲーマーであるという部分と関係していますよね。}

\answer{父・信仁}{ゲーマーが一時期タイピングに挑戦したという感じですから。それなりの結果は出たと思ってるけど、本質はやっぱり、ゲーマーなんだよな、どこまで行っても。「タイパー」って呼ばれたら、それはタイパーでいいんだけどね。}

\answer{あきうめ}{うん。}

\question{ゲームの中の一種類としてタイピングをやっていたと。}

\answer{父・信仁}{「タイパー」って意識したことないよな。}

\answer{あきうめ}{ないな。やってたのはゲームやし(笑)}

\question{ゲームと言えば、タイピングを始める前にはレースゲーム等をやっていたと伺っています。}

\answer{父・信仁}{あれは俺の影響でね。リッジレーサーなどを。}

\answer{あきうめ}{え、他のゲームのことも書いてくれるんですか(笑)}

\question{レースゲームとかをやっていると、コンマ一秒とかの感覚に慣れてきたりとか、タイピングに通ずるものがあるのかなと(笑)}

\answer{あきうめ}{あー……! 確かにタイムアタックとかやってたからな。タイムを競うっていうのはそこから来てるかもな。}

\answer{父・信仁}{幼稚園の頃からやってるわけだからね。}

\answer{あきうめ}{記録の世界みたいな所にそこで慣れていたというのも、あるかもしれないですね。}

\answer{父・信仁}{リッジレーサーのランキング上位の人を家に集めて、というのがそもそものサミットの始まりなんですよ。ランキング一位や二位のすごい人らが来て、その中にまじって、当たり前のようにゲームをやってた。それもただ遊ぶんじゃなくて、競っていた。他のゲームで言うと、鉄拳でもそう。オフ会で対戦をやって、小学校低学年でもバリバリ大人顔負けだった。本質として競うのが好きなんだよな。勝ちたい、という。}

\question{勝負の世界の中には、タイピング以前からいたということですね。}

\answer{父・信仁}{当たり前のように、いましたね。負けん気が強いというのはタイパーならみんなそうだと思いますけど。そうじゃなかったら、そこまで速くなりたいと思わないですから。}

\question{関連しますが、タイピングの世界から離れた後にどういう活動をされていたのかも伺っていいでしょうか。}

\answer{あきうめ}{ゲーマーに戻りましたね(笑)ホンマのゲーマーに戻っちゃった感じです。}

\answer{父・信仁}{ゲーマーですよ。どんだけやるんだよ! ってくらい。}

\answer{あきうめ}{あと、音ゲーを始めた。これはタイピングの影響だったかも。}

\question{音ゲーですか。今どきのタイパーもやっている方多いですね。やはり音ゲーでもガチ勢なんですか。}

\answer{あきうめ}{いやぁ、まだまだガチ勢じゃないです(笑)弐寺\footnote{beatmania IIDX}をやっていて、SP\footnote{筐体の片側 7 キーを使うモード。}は十段ですね。でも SP で全然上達感がなくなっちゃって、DP\footnote{両側 14 キーを使うモード。}に逃げたんです。そうしたら、二週間くらいの練習で九段に届くとこまで行って。こっちの方が行けるんじゃないかと思って、今は DP を専門にやってます。タイピングの経験があるんで、やっぱり全部の指を動かしたくなるんですよね。}

\question{こういう、あきうめさんのタイピング以外の話に興味がある人、多いと思いますね(笑)}

\answer{父・信仁}{彼女はいるんですか、とか?(笑)}

\answer{あきうめ}{ノーコメントで(笑)}

\question{話を戻しまして、ゲーマーとしてのタイピングですが、TOD のお話でも正確性重視のスタイルを強調されていましたね。}

\answer{父・信仁}{根底に「ミスしちゃダメ」というのがあるんです。「タイピングとは何ぞや?」と考えます。「タイプする」という行為は、そもそも何を目的としているかと考えれば……「速く打ちましょう」ではないですよね。スタート地点は必ず「文字を正しく入力する」のはずなんです。キーボードという装置は入力を目的とする道具ですから、正確性があくまで第一、大前提。もちろん一つのジャンルとして「極限の速さ」を競う文化はあっていい。でも、あまりにもそればかり、それ一辺倒では、歪ですよ。極限の正確性を競うような文化・ランキング・ソフトも、もっとあっていいんじゃないかと思います。}

\question{最高記録主義で、正確性が算入されないタイプウェルがメジャーで、対抗できるソフト・文化がない現状がありますからね……。}

\answer{父・信仁}{自分個人の意見としては、タイピングでランキングを作って競う時には、本来はまず正確性を競うべきだと思う。正確性は 100\% で終わりじゃないですか。なので「100\% 以上がないから、そこで終わりじゃん」と言われる。でも、それは違います。逆にそこから始まるんですよ。その中で誰が速いんだ? と、これならわかる。速度にはゴールがなくて、正確性には 100\% という明確なゴールがあるのに、見えているその地点すら目指さないんです。大変さから目を逸らして、逃げているだけだと思います。}

\question{正確性は前提という考え方ですね。}

\answer{父・信仁}{電卓と一緒。電卓は、打って計算した答えが正解で、初めて評価されるわけです。その上で、どれだけ速く計算できますか、という競技になっている。答えがめちゃくちゃでも、速いからすごい、とは絶対にならない。今のタイピング界の考え方からは、その部分が抜けてます。タイピングとは本来、知的な行為だと思うんです。ミスをしないで高速に打つというのは、ピアニストがミスタッチなく名曲を奏でるのと一緒で、芸術の域に達する可能性を秘めている。みんなが正確性に対する意識をもっと持って、目指していけばそういう世界も見えてくる。「ミスをしてもいいからとにかく速さ」というのは、電卓やピアノの世界では通用しないでしょう。正確性 99\% の人のピアノなんて聴いてられないですよ。}

\question{確かに 100\% 正確に演奏できることは前提になっていますね。}

\answer{父・信仁}{もちろん結果的に 100\% で打てたかどうかが重要ではないんです。どんな人間だって絶対 100\% で打てるなんてことはないですから。ピアニストにだってミスタッチはあるでしょう。でも、打つときに正確性を考慮して、100\% を目指そうと思う人間は、たとえそんなに速くなくても「タイピスト」として評価されるべきだと思います。速さだけを追求するぜという人は「タイパー」。この使い分けが個人的に気に入っています。}

\question{「タイピスト」と言えば、毎パソは正確性を重視した競技ですが……下火になってきていますしね。}

\answer{父・信仁}{Pocari さんたにごんさん(といった毎パソに力を注いでいた当時の「タイピスト」)はすごいんです。彼らはタイプウェルでも自然と正確性重視でしたよ。速さだけでいうと誰々だ、という評価しかないと、どんどん記録は塗り変わるから「誰々はオワコン」となる。タイプウェルのスピードだけで言えば、彼らなんて今ではもう全然なんだろうけど、でも実際に大会みたいな場をつくって正確性も含めて一発勝負をしたら、今のタイパーなんてボロ負けしますよ。まったく太刀打ちできないです。}

\question{それは僕個人としても、事実として痛感するところです。}

\answer{父・信仁}{昔の人の中にも正確性を重視してやっていた人はいましたけど、圧倒的なスピード至上主義の勢いに押されて、何の評価もされない。それに権威を持たせるということをやらない限りは、みんなの目はなかなかそっちに行かないんです。限界の速度を出すというのも、もちろん難しいでしょうけど、正確に打つというのも、難しいですよ、と。「ゆっくりでも良いから正確に打ってみ、その代わりミスしたら指ちょん切りますよ」とか言われたら、動けないですよね。その事実から目を逸らさず、意識していかないと。タイムが 0.001 秒でも速ければそれが偉大で凄いという価値観だけでは、どうしてもゲーマーからすると奇異に見えるんです。野球で言えば、アマチュアで 160 キロの速さで玉を投げられて、その代わりコントロールは全然ダメですって奴がいたとして、そんなの絶対プロになれないでしょ。スピードをコントロールできる技術があって初めて、そういう場に出て評価される。}

\question{色々なたとえがうまくはまって驚きます。}

\answer{父・信仁}{100 分の 1 秒、 1000 分の 1 秒なんて動画で見たって、そんなの一緒ですから。動画で見てこっちの方が 100 分の 1 秒速くてスゲーとか、絶対ならない。そういうレベルになった時には、正確性はむしろ目に見える差だし、100 分の 1 秒遅くてもノーミスだったら絶対にそっちが評価されて然るべきです。「この速さでミスしないで打てるのか!」と。……現役でやっている方にこんなことをくどくど言ってしまってすみません。でも、最後の頃はそういうことを主張したくて、色々言ったりやったりしていたんです。TOD2 の動画\footnote{ニコニコ動画 sm2677255}にしてもそう。伝えたかったのはそこなんです。}

\question{あきうめさんとしては、正確性が評価に入らないような競技をどう思いますか。}

\answer{あきうめ}{あまり面白くないと思いますね。ただ、あんまりやっていないので……やっていたのは打鍵トレーナーくらいやな、ミスペナルティがないゲームって。でも打鍵トレーナーでも「ミスしないように」とは心がけてましたね。なので、他の人にも正確性も考えて打って欲しい、というのはあります。}

\question{あきうめ氏が高校に入って少しして、タイピングから離れていかれることになったわけですが、その辺が理由なのでしょうか。}

\answer{あきうめ}{自分としては、別にそれが特にあったわけじゃなかった。}

\question{ゲーマーだというお話からいくと、いいゲームがないから……という感じでしょうか。}

\answer{あきうめ}{TOD のようにゲーム性があって、対戦が出来るようなゲームがまた出れば、やりそうですね。}

\answer{父・信仁}{自分にはやはり、速さだけを求めていくタイピングの世界に行き詰まりを感じたというのが……ありました。俺らの本質は何か? と考えると、ゲーマーだと。TOD から入って、e-typing で正確性込みのスコアアタックをやって……正確性込みのゲームをやってたんです。なので、どこかで区切りをつけようかという話はしていた。タイピングをやっている間はゲームの方が放置状態だったので、そろそろ戻ろうか、と。中学卒業までとみてたんですが、結果的に高校に入った年の夏休みまではやってましたね。e-typing の元気ワードで 810pt を出したのはその最後の夏休みです。}

\question{今に至るまであきうめさんの e-typing 最高記録となっている、置き土産ですね。}

\answer{父・信仁}{今でも、あの元気ワードの記録の WPM がすごい、まだ抜かれていないと、そんな部分ばかりクローズアップされてますよね。でもそんな部分、本質じゃないんです。あれは最後だったから、そういえばまだ 800pt 出してなかったっけ、これが残ってたわと、800pt を出すために、本来のスタイルではない打ち方で、無理矢理出したというだけ。}

\answer{あきうめ}{確かにあの記録は、ゴリ押しで出した感があるなぁ。WPM は 858、ミスは 10 回とかで……かなり多かった。}

\answer{父・信仁}{本来は正確性 100\% のスタイルですから。なので同時期に長文も打って、どこまで来たのか、正確性 100\% を維持したままどれだけ速く打てるのか、その確認をして……やってきたことのケジメをつけて、終わりにしたんです。}

\question{最近公開された e-typing の長文 Error の動画\footnote{ニコニコ動画 sm14896152}がそれですね。}

\answer{父・信仁}{あれは当時としては、記録を登録できるわけでもなし、動画を公開したいわけでもなし、自分らの確認のために打っただけなんです。実際は撮ってたわけですから、公開することはできたんですけどね。}

\question{この頃にはもう、タイパーの中での評価や競争といったことは、眼中になかったのですね。}

\answer{父・信仁}{そうそう、俺ら親子でやってきたことのケジメ、集大成を形にしたかっただけだった。}

\question{主張の詰まった熱いお話をありがとうございました。最後にメッセージなどあれば、頂きたいと思います。}

\answer{父・信仁}{やはり正確性重視の件は、強調しておきたいです。十年一区切りということで今回一度まとめて、また次の十年を見据えて行くとなると……もうそろそろ、そういう転換がないと、どうしようもなくニッチなままで終わってしまうよ、と伝えたい。正確に打つことがどれだけ大変か、意識してください。}

\answer{あきうめ}{同意見です。というか、今のタイピング界の情勢というのが全然わかっていないので、良いことが言えない(笑)自分個人としては、本当にただ、楽しんでいただけなので。}

\question{そういえば「タイピングは好きですか」という質問を預かっていました。}

\answer{あきうめ}{タイピング、好きですよ! 誰かが慣用句 800pt 以上を出してくれるか、TOD Normal 台でカンストを見せてくれたら、もっと好きになると思う(笑)}

\question{ありがとうございました。}
