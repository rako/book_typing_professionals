\section{Weather Typing}

\subsubsection*{モルタルコ 氏}
\noindent Weather Typing 開発者。
そのロビーを通じて多くのタイパーが交流、成長していった。
Weather Typing はネット対戦用のソフトウェアとして、また打ち始めの時間抜きに競う競技として、今なお愛用者は多い。

\question{本日はよろしくお願いします。モルタルコさんは凄腕プログラマなイメージがありますので、技術的なお話など、色々伺いたいと思っています。}

\answer{モルタルコ}{Denasu System のモルタルコです。Weather Typing は本体のバージョンは長いこと止まってますが、当時の状況もふまえてということでお受けしました。よろしくお願いします。}

\question{では、開発当時のことから伺おうかと思います。日記\footnote{モルタルコのプログラマ日記 \url{http://denasu.com/diary/}}を見ていたら「Weather Typing 開発裏」\footnote{\url{http://denasu.com/diary/wt.html}}(以下「開発裏」)というページがあり、開発当初のことがまとまっていました。これを元にお話を頂けますか。}

\answer{モルタルコ}{こういうときのためにこのページを用意しておいてよかったです。}

\question{一番はじめに TOD (the Typing Of the Dead) のお話(2000/12/09 TOD を ISDN\footnote{ADSL 普及以前に主流だったネット接続方式・サービス。通信速度は 64kbps~128kbps で、初期の ADSL よりもかなり遅い。} でやってみる)がありますね。}

\answer{モルタルコ}{TOD はかなりはまってました。Denasu メンバーのぱじ氏とアーケード版、DC 版とやって、PC 版に行きつきました。PC 版の TOD はネットランキング機能が充実していて楽しかったですね。PC 版にはネット対戦機能もついていて、大学の研究室のメンバを取り入れてタイピング対戦とかをしていました。大学の友人じゃない人と対戦もしたいなあ、というのがこの ISDN でやってみよう計画だったと思います。}

\question{しかし結果としては「そもそもネット対戦は無理ということですな」とあります。LAN では大丈夫でも ISDN ではやはりきつかったのでしょうか。}

\answer{モルタルコ}{今ならもっと厳しいゲームでも問題ないので TOD も大丈夫なんでしょうけど、ISDN ではさすがに無理がありました。開発裏にその後「2001/03/10 ネット対戦タイピングなフリーソフトを試す」とありますが、Vector で探してそれっぽいのが 1 つくらいとかだったと思います。でも TOD と比べると……という感じで。}

\question{2001 年ですもんね。時代を考えるとフリーソフトとしてあるだけでもすごいことです。}

\answer{モルタルコ}{確かこの頃は TOD とタイプウェルを使っていて、結構な時間をタイピングに費やしてますね。そして思いつきでタイピングゲームを作ってみようという話に。}

\question{Weather Typing はネット対戦というのが開発動機としてはじめからあった?}

\answer{モルタルコ}{そうですね。研究室で TOD を LAN 対戦していたんですが、途中で互いの点数がバラバラになったり割とバグが多かった気がします。それで TOD を活かしつつネット対戦に特化したソフトを作ろうという話になりました。}

\question{開発開始から「2001/03/25 対戦の同期が取れてまともにゲームができるように」までひと月もかかっていませんね、}

\answer{モルタルコ}{ウェザタイの前から細かいのはいろいろ作っていて、ゲームのフレームワークは既に使い古したものがあったので、その辺で早かったんだと思います。タイピングゲームとネットゲームの大変さを知るのはこのずっと後ですね。}

\question{しかしこの段階で、もうワード自動生成やローマ字自動認識にも取り組まれていたとあります。かなり先見的に思うのですが、TOD を見ていて対抗した部分があるのでしょうか。}

\answer{モルタルコ}{ローマ字自動認識は TOD をかなり意識していますね。当時の市販のタイピングソフトでは入力が固定か最初にカスタマイズというのが多かったですが、TOD でがらっと変わったと思います。}

\question{手動で 「ちょ tyo cho」とかカスタムするソフトが多くありましたね。}

\answer{モルタルコ}{あの TOD の方式は凄いなあ、と思いながら、TOD の入力方式にもいろいろ不満があって。「ん」が「n」1 打鍵でも OK とか。}

\question{確かに TOD の n はかなりクセがあります。}

\answer{モルタルコ}{でも後で自分で作ってみると同じところでものすごく苦労したりして(笑)}

\question{はい、後ろにア行ナ行ヤ行が続くと……とかで面倒ですよね n の処理は。}

\answer{モルタルコ}{w/h さんの tsuikyo\footnote{拙作のブラウザゲーム用にタイピング判定部を提供するライブラリ。}も同じような苦労をしていると思いますが、「っん」とか普通ありえない日本語でバグが出たり。あとは TOD ではカスタマイズが全てバラバラで「ちゃ」は cha と打てても「ちょ」が tyo だったりするのが個人的に気に入らなかったので、変えてみたりですかね。}

\question{当時からかなり力が入っていたんですね。その後のバージョンアップで入ったものもあるんでしょうが、今の Weather Typing では入らなくて困る入力がほぼないように思います。}

\answer{モルタルコ}{やりすぎて「どぅーどぅー」とか公式ワードに入れてしまいましたが。}

\question{dwu-dwu-(笑)}

\answer{モルタルコ}{あとはワードの自動生成も初期バージョンから付けていましたが、最初は TOD みたいに面白いワードを入れようとはしてました。}

\question{やはり TOD の影響はすごくあったと。}

\answer{モルタルコ}{はまりすぎですね(笑)、でも「2 時間チャージ 10 秒キープ」とかワードの面白さでは絶対勝てないなあ、と思い、数日そればっかり考えた末にあのような形に。}

\question{運次第で色々面白いものが出てくるようになっていますね。}

\answer{モルタルコ}{ワードはほとんどぱじ氏が作っているんですけど。初期バージョンで何人かにやってもらったら面白そうな感じだったのでこれでいこうと。}

\question{現在ある各種タイピングソフトで見ても、今なおユニークなシステムだと思います。}

\answer{モルタルコ}{打ちやすさというか実用性を結構犠牲にしてますけど……。}

\question{いや、競技者目線で見てもすごく面白いですよ。タイピングゲームのゲーム性・競技性とワードは切り離せないと思っているんですが、文レベルで固定(e-typing や TOD)というのと、単語単位でバラバラ(タイプウェル)というのとがあって、Weather Typing 方式は見事にその中間。なので長めの部分部分に対するワード慣れと、それをつなぐ能力、どちらも必要となるという。}

\answer{モルタルコ}{そう分析して頂けるとありがたいです。……それで、その辺の実装が終わっていよいよネット対戦を完成させていくのですが、ここからがそうとう苦労してます。ウェザタイはネット対戦としてはそれほど通信量はないんですけど、他のネット対戦と比べるとラグが致命的になったり、DirectX\footnote{Windows 上でゲーム等を開発するためのコンポーネント(部品)群。} の通信機能がブラックボックスでどのポートを使っているのかすら情報が見つからなかったり。}

\question{なんと(笑)}

\answer{モルタルコ}{公開後も安定するまでかなりの時間がかかってますね。その間 IRC とかウェザタイのロビーで毎晩テストプレイしてました。}

\question{やはり通信部分は難しいんですね。}

\answer{モルタルコ}{ネット対戦のノウハウがある人が作ればそこまで苦労はしないと思いますが、元のフレームワーク部分がネット対戦を作りやすい実装になってなかったりで、メッセージを延々送り続けてたりデッドロック\footnote{プログラムの部品同士が、お互いに相手を待つ状態になって、動作が止まってしまう問題。}になったりと苦労しました。}

\question{言語は何で書かれてるんでしたっけ?}

\answer{モルタルコ}{C++\footnote{プログラミング言語のひとつ。} です。}

\question{今みたいにお手軽なゲーム用ライブラリ等が充実してない時代でしょうし、ほとんど手製だったんでしょうか。}

\answer{モルタルコ}{DirectX と画像の読み込み以外は全て自作ですね。その分 Android\footnote{最近流行のスマートフォン用 OS のひとつ。} 版移植は楽だった……のかなあ。}

\question{Android ということは Java\footnote{プログラミング言語のひとつ。} ですね。今時の環境は色々と気が利いてて、こっちが書かなきゃいけない部分どんどん減ってますよね。}

\answer{モルタルコ}{ですね。当時から書きためていたライブラリ群のほとんどが Java とか C\#\footnote{プログラミング言語のひとつ。} では標準ライブラリ\footnote{プログラミング言語のパッケージに添付されてくる、最も標準で基本的な部品。}で用意されたりします。}

\question{開発者として、一昔前の、低レベルな API だけ叩いて丸ごと作るというのには憧れます。組み込み系業務だと今でもそうなんでしょうが……と、話がやや逸れました。ちょっと話を戻してしまうんですが、はじめの開発動機の段階で TOD をあげておられましたが、モルタルコさんは美佳タイプを利用した Typing Attack などの当時の競技タイピング文化とは直接の関係はなかったのでしょうか。}

\answer{モルタルコ}{美佳タイプなど有名どころは試していますが、TOD までは競技タイピングには関わってはいませんでしたね。}

\question{なるほど。しかし結果的に Weather Typing の登場は彼らの文化を非常に加速させることになりました。特にロビーの存在が大きかったと聞いています。}

\answer{モルタルコ}{いろいろ課題はありますが、ロビー自体はあってよかったと思います。私自身は最初ロビーを作るというのは考えてなかったのですが、ネットゲーマーでもあるぱじ氏が絶対必要と言ってきて。とりあえず身内で使えればいいかな、というので開発し、しばらくは開発者のテストプレイ場になってたのですが……公開してみたら、当時掲示板に書き込んでくれてた方達が参加してくれて。ピークのときは何十人も来ていたこともありました。当時 TV でタイピングが取り上げられてたときですね。}

\question{グッジョブ\footnote{TBS 系列で放送されていた「タモリのグッジョブ! 胸張ってこの仕事」に Pocari 氏をはじめ当時のタイパーが出演した。}時代ですね。}

\answer{モルタルコ}{はい。ロビーで実況してました。}

\question{バージョン履歴を見ると「2002/06/11 Ver1.6」でロビー対応とありますね。さすがにロビーが公開されたのは本体の公開からさらに一年ほど後だったんですね。}

\answer{モルタルコ}{そうですね。公開当初はこれだけ盛り上がると思ってなかったので。タイプウェルのトップページやシャドールームさんで取り上げて頂いて、一気に知名度が増えた頃ですかね。TOD の頃から名前だけは知っていた方とかもロビーに来られて、私自身としてもいい機会を持つことができました。でもロビーは技術的にはあまり見るところはないですかね。普通のチャットソフトで。}

\question{レベル(タイピング速度の指標)が表示されるあたり、特化していて面白いなぁと思いますよ。そういえば、今後ロビーサーバを止められてしまう可能性などはあるのか……と気になるのですがどうでしょう。}

\answer{モルタルコ}{今のところサーバを止める可能性は低くなってます。ちょっと前まではずっと自宅サーバを間借りしていたんですが、今はレンタルサーバで運営しているので気楽にやっています。計画停電の心配も少なくなりましたし。}

\question{とてもありがたいです。当時ほど常時賑わうというわけでなくても、依然として数少ないネット対戦できる環境として重宝されていると思うので。}

\answer{モルタルコ}{今後もっと活用できるようになるといいんですが。}

\question{ロビー付きネット対戦の環境が整っているタイピングゲームは今でもほとんどないので、需要は今なおすごくあると思います。ただ、今の若い人は Twitter や mixi や Skype ……というイマドキな環境でコミュニティが形成されているのが普通なので、それとは別にソフト毎のロビーに常駐……というところまで手が回りにくいのかもしれません。}

\answer{モルタルコ}{そういうのもあるのかも知れませんね。逆にその辺とうまく連携するようなタイピングゲームがあれば……もうあるのかな。}

\question{ちょっとしたものであれば、Twitter や mixi のアカウントを使うようなタイピングゲームはありますね。}

\answer{モルタルコ}{話がそれますが、ブラウザ上で対戦するタイプのはどうやってるのか気になってます。Android 対戦とかやろうとすると P2P\footnote{ユーザ間で直接、通信を行う方式。} じゃなくてどうしてもクライアントサーバ方式\footnote{ユーザ間で通信を行いたい場合にサーバを経由する方式。}も、になってしまうので、そんな大規模なサーバ用意できないなあ、とか。}

\question{ゲームとして実用的なフレームレートを出しつつクラサバ\footnote{クライアント・サーバ方式のこと。}でソケット通信するのはかなり大変だと感じます。その辺は「小規模だから」と割り切っているものが多いように見えますね。バトタイなんかはそうなんじゃないでしょうか。実際、負荷が集中するとすぐサーバが重くなり、落ちてしまうこともあります。}

\answer{モルタルコ}{ちょうど今その辺を研究しているところです。タイピングの前にぱじ氏がネット対戦脱出ゲームを作ってるかも知れませんが(笑)タイピングの方も Android vs PC とか考えてはいます。}

\question{Android 版 Weather Typing、一発モノなのかなぁ等と失礼にも思っていましたが、展望を考えられてるんですね。}

\answer{モルタルコ}{まあできたらいいなあ、程度ですけどね(笑)1 年前、大学の先輩に Android vs iPhone があったら面白いんじゃない? と言われて、そこからプロジェクトが立ち上がりました。}

\question{iPhone 版に期待がかかりますね(笑)そういえば、大学と仰いましたが、Denasu System という団体はそういうつながりでできているんですか?}

\answer{モルタルコ}{Denasu System は大学より前からの集まりですね。昔マイコン BASIC マガジンというプログラミング雑誌があったんですが、元々はそこに投稿したりしてました。}

\question{そうでしたか。ベーマガ\footnote{マイコン BASIC マガジンの通称。}……世代がひとつふたつ下ですけど、伝聞で存じてはいます。}

\answer{モルタルコ}{もう休刊してかなり経ちますからね。プログラミングする環境とか情報は今の方が整ってますけど、初心者が入っていける環境ではなくなってきてますよね。}

\question{分野内での分化・多様化が進みすぎて、情報共有も難しくなっていたりと、課題はちらほら見えますよね……。っと、気づくとまた話題が(笑)Weather Typing に戻しまして、ちょっとマニアックなあたりの話を伺いたいです。Weather Typing の面白い部分として、Dvorak や NICOLA にソフトレベルで対応している点があります。JIS かなはまだわかりますが、Dvorak や NICOLA にも対応。これはどういう経緯で?}

\answer{モルタルコ}{最初のバージョンはどうでしたっけ……確か掲示板で意見を伺っていくうちに入力方式が増えていったと記憶しています。}

\question{開発履歴を参照しますと「2002/01/06 Ver1.4」で「かな入力、NICOLA 配列、50 音配列、dvorak 配列をサポート」とあります。JIS かなと同時にこれらが一気にサポートってすごい(笑)}

\answer{モルタルコ}{ウェザタイは対戦なので、プレイヤーが一番得意な打ち方で対戦したいというのがあって。自然にローマ字 VS かな入力までは作っていたはずです。}

\question{実はタイパーの間に Dvorak や NICOLA の存在を知らしめた原因のひとつではないかと勝手に思っているんですが。}

\answer{モルタルコ}{え、そうなんですかね。}

\question{入力方式に「Dvorak」とかあると、やはりなんだこれと気になりますし、調べますよね。}

\answer{モルタルコ}{そうですね。NICOLA は結構練習しました。Dvorak はあまりやっていません。かなり調査はしましたけど、入力方式ごとにネットランキングがあったらどの入力方式が一番なのか? って盛り上がるんじゃないかというのはあったと思います。それは Android 版のスコアをあえて同じランキングに入れているのと同じ理由です。}

\question{当時から配列間での対決というのを視野に入れてらしたんですね。最近のタイパーの間では割と配列話も認知されていて、タイプウェルを打つのに特化した配列、とかマニアックなものが開発されて競われてますが、実はそういう文化にもつながっているのかもしれません。}

\answer{モルタルコ}{マクロとかじゃなくて配列で対応、ですか。}

\question{ええ、具体的には TWW 配列、稲配列などが実使用者のいるタイプウェル特化配列ですね。タイプウェルのランキングにも参考記録としてそういう特化配列での参加も認められたりし、市民権を得た感があります。}

\answer{モルタルコ}{その辺は管理人さんメモにありましたね。}

\question{普通の Qwerty、JIS かなと、そういう特化したものを同じ土台で競わせていいのか、不公平感は出ないか、など議論も一部でヒートしましたが、Weather Typing のランキングではとうの昔に通り過ぎた地点だったようですね(笑)}

\answer{モルタルコ}{ウェザタイは基本的に要望が強ければソフト側で対応、ですね。構造上フックも入れづらいでしょうし。}

\question{入力方式のプラグイン化\footnote{ソフト本体の外部のファイルなどで、機能を拡張できるようにすること。}は難しいですよね。Dvorak みたいなキー単位のリマップ\footnote{物理的なキーに対応させる文字を再配置する操作。}がある程度ならいいですけど、一般化すると……。}

\answer{モルタルコ}{そうですね。プラグイン化はちょっと考えてましたが、作る側がすごく大変になるのでやってないですね。実現したらモールス信号入力とかでてきたりして面白いかも知れませんが(笑)}

\question{現実的には外部のキーレイアウトカスタマイズソフトウェアに Qwerty や JIS かなの信号に変換してもらって、と巷ではやっていますね。}

\answer{モルタルコ}{ウェザタイではキーレイアウトカスタマイズソフトウェアがことごとく効かないですね。}

\question{そういえばそうでした。DirectInput\footnote{先に出た DirectX のうちの一部で、入力等を担うもの。} とか、低レベルなところでキー入力を取ってるからですか?}

\answer{モルタルコ}{です。Window メッセージを使ってないためです。}

\question{不正対策と考えると、良い面もありますね。}

\answer{モルタルコ}{Window メッセージだと、マクロ\footnote{ゲームなどで、人間ではなく機械が自動的に操作を行ってしまうようなもの。}組まれたらしょうがない、的なものはどうしてもありますよね。}

\question{はい。タイピングゲームってすごく不正がしやすいジャンルに思います。}

\answer{モルタルコ}{そういう意味ではランキングは目安にしかならないかなあ、と。}

\question{Weather Typing のランキングは健全に見えますよ。}

\answer{モルタルコ}{ほとんど何もメンテナンスしてませんが(笑)}

\question{技術話つながりですと、入力の判定方法やソフトウェア開発に関連して、もう少し伺えればと思います。}

\answer{モルタルコ}{判定はやっぱりオートマトン\footnote{情報系の学問的な概念で、入力を順に受け取ることで、それまでの入力により決定される「状態」を変化させ、何か意味のある出力を得るようなシステム。}ですよね。}

\question{アカデミックにもオートマトン使うのが普通っぽいですね。tsuikyo は、オートマトンモデルと言いつつ、実装としては普通にアレイ(配列)とかつかっているんですけど。}

\answer{モルタルコ}{ウェザタイは \url{http://denasu.com/diary/diary201009.html} にある通りに素直に作ってますね。}

\question{中でもちゃんとオートマトン作って処理してるんですね。すごい。}

\answer{モルタルコ}{処理速度を気にするほどのこともないですしね。}

\question{tsuikyo は JavaScript\footnote{プログラミング言語のひとつ。} な都合で大まじめに構造化してしまうと速度的にネックになりかねないので、と控えてしまった覚えがあります。}

\answer{モルタルコ}{その辺を気にしないといけなくなるんですね。この辺のライブラリ化とか、ウェザタイでちょっとやってますけど、ワードサーバ化\footnote{ワードをソフト本体に組み込んでしまうのではなく、別にワードだけをやりとりするようなソフト・機能を作って、それを経由してワードを提供してもらうようにすること。}とか組み合わせて簡単にタイピングソフトを作れると面白いかも、みたいなのは思います。けどなかなか難しいですね……。}

\question{ワードサーバと入力判定エンジンが便利に存在していると、タイピングゲーム開発は相当省力化できますよね。}

\answer{モルタルコ}{他に放送禁止ワード判定サーバが欲しいです。Google とか放送禁止ワード判定サーバを作ってくれませんかね(笑)間違ったら誰に責任があるのかとか、なかなか難しそうですが。}

\question{ユーザが誰でもワードを登録できたり、機械的にワードを拾ってきたりするものを作ろうとすると、欲しいですね。……さて、まだまだ開発関係でお話が伺えそうなところですが、そろそろお時間となりました。最後に、何か一言メッセージをお願いします。}

\answer{モルタルコ}{Weather Typing は、元は自分たちで対戦をしたくて作ったソフトウェアですが、いろんな人に使って頂いて、意見を伺ったり交流ができて、いい経験になりました。今はバージョンアップがしばらく止まっていますが、今後もロビーやランキングの運営は続けていきます。Android への本格対応、他のプラットフォームなど新しいこともやっていこうと思いますので、これからもよろしくお願いします。}

\question{ありがとうございました。}
