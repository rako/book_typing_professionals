\section{Doping - ネットタイパー文化と現代}

\subsubsection*{俺◆gt4Uu5qyB2 氏}
\noindent 匿名勢としてネットを中心に活動するカリスマタイパー。
リポビタン D 等でカフェインを摂取し集中力を高めて記録を叩き出すドーピングが有名。
人類初のタイプウェル国語 R 総合 ZG 到達、e-typing 817pt 等の競技経歴を持つ。

\question{ではまず、どういうタイパーなのか、来歴といいますか、自己紹介と言いますか、そのあたりから伺えればと思います。}

\answer{俺}{ちょっとこれ見てくださいよ……からあげ弁当っすよコレ。大盛り大盛り大盛りって大盛りシール3つも貼られてますよコレ。}

\question{からあげ弁当の話から入っちゃった(笑)}

\answer{俺}{ご飯の量絶対食いきれないっていう……。あ「からあげ」って人いますもんねタイパーに。\footnote{確かに Twitter やタイプウェルランキングにおられます。}……自己紹介か。好き勝手やってただけのタイパーですって感じですよね(笑)}

\question{蛆虫編(俺◆gt4Uu5qyB2 氏の日記 \footnote{\url{http://www.mypress.jp/v2\_writers/oreoreoreore/idx/}})は残っているので 2004 年から最盛期 2006 年頃までのログは辿れますね。それ以前のことはどうでしょう。}

\answer{俺}{最初、ラウンジにタイピングのスレがあって。e-typing やったんすよ。二本指で。150pt くらい。でなんかムカツクなーと思って、しばらく練習して……299pt くらいになった。それからって感じすね。}

\question{本当に 2ch のスレが最初のきっかけだったんですね。}

\answer{俺}{そっすね。ホント全然あきうめも Pocari も知らなかったっすよ。そん時ちょうどテレビ放送\footnote{2002-2003年にかけて放送された、「タモリのグッジョブ!」。当時の有名タイパーが出演し、この時期タイピング界は大変盛り上がった。}やってたらしいんすけど、自分は全然(知らなかった)っていう。まあその後、上を目指そうっていうモチベを保てたのはあきうめの存在がデカいかもしんない。探せば最々初期の書き込みとかは残ってるんすよ実は。タイピング始めた頃のやつが。最初は確かラウンジに行ったんですけど、後からクラウンてのもできて、そっち書いてたりしたんすよね。最初の方の書き込みだとすげぇ中学生みたいな感じなんすよ(笑)いや実際中学生くらいだったんすけど年齢も(笑)}

\question{同年代なんでわかります(笑)}

\answer{俺}{クラウンの最初のスレで、84 って人がいて、その人がやたら格好良かったんすよ。400pt 出そうっつって気合い入れまくってるだけの人なんすけど……根性で打ちまくってて。あータイピングは根性なんだなって思って、そこからタイピングに関して微妙に精神論派になったんすよね。}

\question{結構、精神論派ですよね。}

\answer{俺}{いやー最近精神論が衰えちゃって、技術は増しても記録下がっちゃってるみたいな。あ、最初に俺がクラウンに書き込んだのこのスレっす。\footnote{\url{http://www.geocities.jp/iris\_haisure/type2ch/thread24.html}}}

\begin{itembox}[l]{タイピング@クラウン}
\begin{spacing}{0.7}
{\footnotesize
\begin{verbatim}
947 名前:俺 ◆ADxkdMqw
投稿日:03/10/28 08:08 ID:A035OLQ2
 最近タイピングを始めました
 pc一般のスレを見て84さんのファンになりま

 84様を心の師匠として頑張りま
\end{verbatim}
}
\end{spacing}
\vspace{-2mm}
\end{itembox}

\answer{俺}{最初に書いたのがこの 947 番っすね。「頑張ります」って言い切らないようにしてるあたりからもうセコさが見えてくる(笑)}

\question{今で言う「~ますん」みたいなやつですね(笑)}

\answer{俺}{このあたりでタッチタイピングから覚え始めた~みたいな感じなんすよ。11 月の時点で元気ワード 323pt とか出してますね。}

\begin{screen}
\begin{spacing}{0.7}
{\footnotesize
\begin{verbatim}
964 名前:俺 ◆ADxkdMqw
投稿日:03/11/05 18:34 ID:LUahOCUA
 >>959
 まず屍になってもらわないと……
 元気の出る言葉デ323
 秋ワードで305位
 いまはwpm上げるために必死に一定のリズムで叩き続けて
 だんだんペース上げていく特訓をシテオリマス
 今まで332が最高だった罠
\end{verbatim}
}
\end{spacing}
\vspace{-1mm}
\end{screen}

\answer{俺}{この頃から「必死に一定のリズムで」とか訳わからないこと抜かしてますね。でもしろこさんも似たようなことやってましたよね。}

\question{この頃から「俺」ってハンドルなんですね。}

\answer{俺}{メルマガ板っていうすっげー過疎な板があって(笑)そこに一人でスレ立てて、1 から自分の書き込みだけで 1000 目指してたんすよね。「俺」って名前で。今でもまだあるのかな……。……あった。見つかるもんなんすねー意外と(笑)\\

ひっそりと俺が1000目指すスレその1\footnote{\url{http://logsoku.com/thread/pc.2ch.net/mmag/1039432601/}}
}

\answer{俺}{そこでコツコツ書いてた頃だったんで、タイピング関係のスレに行くときも「俺」だったっていう。このスレもなんか途中からラウンジの連中がやってきたりして、色々書き込まれたり、俺の真似して一人で 1000 目指すスレが立てられたりして。}

\question{謎のカリスマ性はこの時代から既に発揮されていたと(笑)}

\answer{俺}{なんかついてきてましたね。で、タイピングもこの頃から割と人を抜き始めていて、割と人を萎えさせてましたね。HUNTER×HUNTER のハンター試験でキルアが勝った時みたいなコピペ貼られて(笑)時速 100 キロで走っているパンピーを時速 300 キロで抜かしていったーとかいう。そういう(抜かれて萎えるような)人がいることを知らなくて。こんな人がいるのかーと俺もちょっと萎えたっていう。俺伸びたのがそんな気に入らなかったんかー、みたいな。今はもう遠慮する必要ないですけど一時期(タイプウェル国語Rでトップだった頃)は結構遠慮してましたね、ランキングとかで。別に大して一位取りたくもないのに取っちゃったら、本当に取りたいやつがかわいそうかなー、みたいな、すっげー最悪な感じのアレで。勝ちたいやつがいるんだったらそいつに譲ってもいいかなー程度の気持ちでした。}

\question{成長速度には定評がありますよね。}

\answer{俺}{半年で XA まで行くあたりは結構速い方だと。そっから上は辿り着いてる人がいなかったから最速って言われてただけみたいな。テルさんも最速 ZF なだけじゃないすか。言ってしまえば。}

\begin{screen}
\begin{spacing}{0.6}
{\footnotesize
\begin{verbatim}
19:俺 ◆ADxkdMqw
:04/03/10 20:56 ID:haDy07X6
 新スレ初記録更新書き込み
 基本常用語XCから一気にXAに爆発更新
 キタ──────(゜∀゜)───────!!
 ついでに総合XBも
 キタ━━━━━━(OWO)━━━━━━━!!
 久し振りの爆発更新&総合XBなんでマジ嬉しい、ヒャッホウ!
\end{verbatim}
}
\end{spacing}
\vspace{-1mm}
\end{screen}

\answer{俺}{まだこの頃は、天狗になってなかったんすよ(笑)総合 XS になったときにふざけてラウンジで天狗になったらめっちゃ叩かれましたからね(笑)「総合 XS 来たぞ! 雑魚共が!」って書いてボコボコにされました(笑)逆に Z に入ると俺まだまだだなって気分になってた。}

\question{この頃からキーボードは Realforce ですかね。}

\answer{俺}{多分そうっすよ。e-typing 400pt になった辺りで買ったんで。で、この辺りからもう調子乗って蘊蓄を語りだしてますね。}

\begin{screen}
\begin{spacing}{0.6}
{\footnotesize
\begin{verbatim}
221:俺 ◆ADxkdMqw
:04/06/02 03:49 ID:nSsSq6jM
 XXへの道がなんとなく見えてきたような気がした。
 打鍵は「強く速く」ではなく「軽く速く」というのがいいことに気づいた。
 machineに行くだけなら、打つ速さがそれなりにあれば良いけれど
 それ以上を目指すには、打つ早さは勿論、打ってからの次の動作が
 重要だと思った。
\end{verbatim}
}
\end{spacing}
\vspace{-1mm}
\end{screen}

\answer{俺}{これ書き込みが 2 日で、26 日に XX 出してますね。}

\question{ひと月かかっていないと。}

\answer{俺}{さらに一年経ったらもう完全に、完全に天狗になってます(笑)}

\begin{screen}
\begin{spacing}{0.6}
{\footnotesize
\begin{verbatim}
898:俺 ◆gt4Uu5qyB2
:05/05/31 21:56 ID:ktJUMNHI
 5月末までに500いけなかったら引退しろ
\end{verbatim}
}
\end{spacing}
\vspace{-1mm}
\end{screen}

\answer{俺}{バトタイとかで勝ちまくるようになってて、天狗になったんじゃないすかね。この頃はまだタイピングに対する熱はすごくあった感じ。今だったらこんな事言わないもん。「すごいねがんばれば伸びるよ」くらいで終わっちゃう。}

\question{この頃から先の話は「蛆虫編」に割と詳細に残っていますね。TWJR 総合 ZH、常用 ZG 到達、nepocot 名義で当時 TWJR ランキングトップであった RGB-1011 (o-ck) 氏を抜いてトップに。直後ルール違反ということでタイプウェル公式ランキングから削除されるも、「最終更新」とか「帰らせてもらうからな!」とか言いつつ記録を伸ばし続け、2006 年 7 月、非公式ながら初の総合 ZG 到達。}

\answer{俺}{そっすねー、GANGAS 削除の影響もありましたけど、あの頃はラウンジで煽りまくって誰か(自分を追って)来ないかなーと思ってたんですけど、結局誰も来なかったのでつまんなくなって最終的にやめちゃったという感じ。記録は(勃起教)教祖さんが地味に、ぽいって俺を超えていきましたけどね。あっさり。ぱないっす。あのとき(トップレベル勢でライバルであった)教祖と RGB と両方とも独自運指っすよ。ばりばりの我流の連中。後からテルさんが来たおかげでちょっと標準運指派が盛り返して来てますけど。}

\question{確かに、彼らのスタイルで圧倒されてしまうと標準運指派としてはちょっと複雑な気分になるかもしれません。}

\answer{俺}{相手が標準運指だとまだわかりますもんね、気持ち的に。抜けるんじゃねーかなって。でも我流だと変な所で加速するみたいな。そんな打ち方しねーとその速度は出ないわみたいな。今思うと変な所で加速するというより、指多めに使えるようにして遅いところを潰すって方が正しい感じですけど。タイプウェルにカナ別のタイム集計出せるやつあるじゃないですか。独自運指の人で速い人ってあれが平坦に近いんじゃないかなと。標準運指寄りの俺とかが遅い部分が、そんなに遅くないんすよ。一番遅いかなでもそんなに遅くない。で独自運指やっぱり強いのかなぁと思ったり。でも(標準運指である)テルさんが圧倒的だからまだいっかーみたいな。}

\question{はい、テルさんが無双している間は標準運指派にも希望が(笑)}

\answer{俺}{テルさんは、もう later\footnote{ひろりんご、ひろ?±赤い糸、XψLaterχ}氏と同格くらいになりましたもんね。速度的に。毎パソ初見、ゴリ押しで 1100 文字とか行けるくらいですよね、多分。later テルさん二人、1100 初見行きますよ多分。初見だと俺良くて 900 とかですから。漢字含有率にもよりますけど。でもタイピングって二位くらいが一番楽しいですよ。上がいて。テルさんみたいに一人で伸ばしていけるタイプならいいですけど。}

\question{本当に一位になってしまった時って、どうすればいいのかちょっと困りそうですよね。たにごん氏も最大 15 種目でトップになって、TWEW でも無双していて、全然抜いてくるやつがいないからもういいかな、となってしまったと仰っていて。}

\answer{俺}{そうっすよ。まだ打っていいのかなーみたいな。やっぱ競って、競った結果、人の記録を抜くのが楽しかったんじゃないですかね。一時期、競争心じゃなくて向上心の方にシフトしようと思って、結果全然伸びなくなった時期があったんすよね。でやっぱ競争派なんだなぁと。最近は教祖さんだけじゃなくしろこさんやテルさんやダニーさん\footnote{いずれも、俺氏のタイプウェル国語R基本常用語のタイムを抜いているトップレベルタイパー。}が出てきたんで、やっと楽しくなってきて、また本気で打とうかなーとも思ってたんですけどね。まあ GANGAS で削除されてなかったらまだやめられなくて打ってたかもしんないっす。}

\question{その一件、有名ですけど、行き違いがあったと伺ってます。}

\answer{俺}{まさか GANGAS からメールしてくるとは思ってなくて。あちらから「削除していいんですか」みたいなメールが来てたらしいんですけど、受信とかしないメールボックスだったのでそんなの知らなくて。それで無視する形になっちゃったのがまずかったらしい。絶対 GANGAS ブラックリスト入りしてますよ俺(笑)}

\question{その削除の一件の直後の、「凡人的最終記録」に向けての怒濤の更新。あれについて少し伺いたいです。}

\answer{俺}{えーと、あの記録の常用語 22.6 って、公式に破られた\footnote{しろこ氏によって。}のは去年 (2010 年)の 6 月だったんだ。実は割と最近ですね。常用以外は多分そんなやり込んでなかったんで。}

\question{とはいっても当時の日記の記述を見るに、半端ではないやり込みがあったように見えますよ。}

\answer{俺}{「当時としては」みたいな(笑)テルさんのやり込みに比べたら……ってことですけど。でカタカナは 24.713 で、これはすぐ破られてますね\footnote{RGB-1011=o-ck 氏によって。}。漢字は俺 24.067 で、これ破られたのはホント最近だな、2011 年 2 月\footnote{テル氏によって。}。漢字はテルさん ZF も出しそうですけどね、何より慣こと 22.8 ってのが半端なさすぎるんで。当時、総合 ZG 行けるんじゃねーかと思ってずっと粘ってたんですよ、ガーッってやって。確か 1 週間とか 2 週間とか、リポ D 飲んでずっと。まあ拡張は多分、やり続けてたらもっと伸ばせた……と思うんすけど、休み入れないとなかなか。あれやった後、打ち過ぎて完全に腱鞘炎なりましたからね。}

\question{こうして今見返しても信じられないくらい記録が詰まりに詰まった 2 週間ですね。最近になるまで公式には抜かれなかったような記録がすべてその期間中に一気に出されたという。}

\answer{俺}{毎回毎回覚醒したような感覚で打ってましたからね。覚醒しつつ、しかもその流れに介入できたんすよ。覚醒してて、めちゃくちゃ打てて、指動いておー速ぇーってなる時あるじゃないですか、あの時に、さらにもう一歩、こうしたら速いんじゃないかみたいな感じで、切り込めていたんすよね。覚醒の中にさらに突っ込めたんすよ。それって多分クスリ……カフェインで集中できてたからっていう。そういう風に使ってましたねリポ D を。あの時みたいなのもう無いですからね、まず覚醒すらしないっていう。}

\question{なるほど。他の活躍として印象に残っているのは「ドヴォラ記」(2008 年にぽぷら氏が中心となって行われた、複数のタイパーで Dvorak 配列を習得しようというプロジェクト。\footnote{\url{http://d.hatena.ne.jp/spcia/}})ですね。}

\answer{俺}{あー、あれは相当必死でしたね。}

\question{俺さんは 2 月 14 日、バレンタインデーですね(笑)から始めて 2 月 20 日まで、ちょうど一週間で、配列を覚えるところから、総合 XJ、基本 1500 は XH まで到達。これは配列習得 RTA じゃ全一間違いなしだと思います。}

\answer{俺}{今のところそんな感じっすね。多分 boraboru さんあたりがなんかやったら絶対もっと行ってるような気がする。ちゃんだに(ダニー、DANNY)さんもあっという間に TWW 覚えてたし。でも(あの Dvorak 習得速度は)今んとこでは一番マシじゃないすかね。当時ぽぷらも半端なかったですけど。}

\question{そうですね、ぽぷらさんも必死に対抗されてました。けど、最終日付近での俺さんの爆発がちょっと圧倒的すぎましたね。}

\answer{俺}{しかもあれハンデついてましたよね、1 日か 2 日か俺より早く始めてたっていう。}

\question{俺さんの開始初日では、端から見て埋まりそうにもない差があるように見えたんですけど……一気に最後の二日で抜いていきましたね。最終日に一気に総合 SE から XJ まで持っていっています。}

\answer{俺}{なんだろ、やっぱ S に乗るまでが長かった。ちゃんと覚えて打てるようになるまで。}

\question{タイピングの勝負で、他の方が「よーし Dvorak 頑張るぞ」と和気藹々している中、「総合 SS 勝負、負けたら死にます」と言ってしまうあのインパクト。意識が完全に別次元だったと思います。}

\answer{俺}{ほんと必死ですよ。あの頃は宣言したこと大体実現できてましたからね。宣言して無理だったのは DAYDREAM さんに勝てなかったとかそれくらいですね。当時の彼が総合 ZI の 7000 だか 6000 くらいで、俺も数千 pt は更新したんですけど、あと 1000pt くらい足りなかった。その流れで Pocari とか抜いてはいたんですけどね確か。目標を設定してガンガンやるっていう。目標って言うか「予定」とか言ってましたからね。目標のことを「予定」と。わけわかんなかったっすね。}

\question{さすがでした。後はマニア寄りな話になりますけど、TWW 配列\footnote{タイプウェル国語Kを高速に打つために作られた、中指シフトかな配列。俺氏自身のQwertyでの競技タイピングの経験に基づいて設計されている。}についてちょっと伺いたいです。}

\answer{俺}{えーと JIS かな打つの、覚えるのめんどくさくって、中指シフトやろうと思って、一応やってみたけど愛着湧かなくて。全然やる気起きなくて。自分で作ったらやるんじゃねーかなーと。でどうせタイプウェルしか打たないから名前 TWW(タイプウェルウェル)にしよう、みたいな。で適当にやってたら制作者なのに抜かれまくって、今だと TWW 全国 4 位とかっすよ(笑)今ではちょっとクソ配列だなぁと思ってきたりしていて。あれで本当に XB とか出せるのが俺は信じられないんだけど(笑)ゆっくり順調に打ってる間はいいんですけど、加速しようとすると結構片手に負担が連続して来たり。あとは単純に中指シフトの構造上の問題、表裏が入れ替わっちゃう混乱とかですね。}

\question{中指シフトは速くなると結局中指がネックになってくるよねという話を聞きます。}

\answer{俺}{と言っても、中指以外はないしね……それ以上の指って。それがシフトゲーの限界なのかな。TWW は作り直そうかなぁともちょっと思ってるんすよね。もう常用語特化にしちゃおうかなと。それかカタカナ。慣用句はやんない(笑)}

\question{「慣用句ことわざ」に特化した配列って難しいですね。あれが速いってことは、多分日本語一般、かなり広く速いんで。}

\answer{俺}{あれは多分中指シフトじゃないやり方じゃないとダメだと思う。なんだろ、四段目の打ちやすいとこだけ使うみたいな。でも作るの難しいから結局三段に落ち着くみたいな。}

\question{結構三段と四段の壁というのはありますね。作るのもそうですが、JIS かなを相当やり込んでいる方とかでないと、後から四段配列をぽんと使って速いというのはなかなか難しそうです。}

\answer{俺}{あ、からあげ弁当食ってもいいっすか? 食えなくて放置されてるっていう……。}

\question{あ、すいません、どうぞどうぞ、食べて下さい。}

\answer{俺}{……とりあえずチンしてきました。あれ多分チンし終わってもおいといたら暖めといてくれるやつなんで。他に何かあります?}

\question{リアル寄りではなくて、匿名文化、2ch とかを中心に活動されていた方なんで、そういう立場からみた文化的な話を伺えたらと思うのですが。}

\answer{俺}{えーと、そういや表タイピング界とか裏タイピング界とか 2ch のスレで言われてましたけど、それってこういうことなんじゃないかなと思うんすよ。
\begin{itemize}
 \item 表タイピング:毎パソ勢
 \item 裏タイピング:ネットタイパー勢
 \item 魔界タイピング:謎の連中 史上最速とかw
\end{itemize}
}

\answer{俺}{ネットタイパー勢って地味で、一般世界で名前出しても通じないじゃないですか。でも毎パソは書けますもんね、履歴書とかに。一時期俺は魔界勢でしたけど(笑)勃起さんとかも魔界ですよ。サブマリン系が。そして、しろこさんが今魔界入りしたっていう。}

\question{魔界入り(笑)}

\answer{俺}{魔界は基本速いのしか出てこないですから(笑)速いやつしかまず名乗らないし、広まらない。史上最速とか魔界過ぎますよね。タイプウェルどんだけなんだよ、って訊いても全然あいつ答えないですからね。エルサレム、こじみんもギリギリ魔界っすよね。誰々説とかありますけど。あーそうだそうだ、この機会に、俺が総合 ZG になった時に、俺より上だなーって思ってた人を言っとこう。}

\question{そういえば才能あるやつがいるとか、もっと上を知ってるとか、当時の日記にありましたね。}

\answer{俺}{まず later とあきうめなんですよね。あと実用初見のぷんだ、実用 Pocari もやばいなと思ってましたね。総合 ZG の時にも勝てないかもなと。やり込んだらベース速度で勝負にはなるかなと思ってましたけど……あとは英語の dqmaniac のノーミスっぷりハンパねぇなみたいな。大体本番でノーミスじゃないすか。ノーミスの難易度って英語の方が上ですよね。あと何をしてでも勝つみたいな意気込みが凄かった時期とか普通に尊敬してましたよ。なもんでゲームで ZG になっても、毎パソの本当に上位陣にはある程度、一目置いてましたね。}

\question{なるほど、毎パソのような表タイパー勢とはやや距離を置いて競技に打ち込んでいた俺さんからそういう話が聞けるのは興味深いです。}

\answer{俺}{彼ら表勢は割と人格者っすよね。逆に俺はネットタイパー勢の中でも割とクソなキャラみたいな感じですもん(笑)やってた当時はみんな馬鹿にしまくってましたからね。}

\question{あれは演技入ってるんだろうなと思いつつ、なかなか強烈でした。}

\answer{俺}{演出込みですけど、割と本気なとこはありましたよ。なんつーか、ああいう人格の部分まで、タイピング速くなるように、意識的に書き換える的なことをやってた。バトタイとかで格下に 1pt でも取られたら自分を罵りまくってましたからねチャットで。「こんなクソにカスにゴミに 1pt 取られるなんて俺ももう終わりかよ」みたいな(笑)}

\question{そんな人とは打ちたくないですホント(笑)}

\answer{俺}{そういうとこまで込みで勝つこと考えてたんで。だから「負けたら死ぬ」とか言い出すんすよ(笑)才能ない雑魚が、勝ち残るために必死だったっていう。凡人でもあそこまで何でもやれば、ある程度行けるんだなみたいな。でもそういう凡人のがんばりって母数が増えれば増えるほど埋もれていきますからね。天才肌の奴とか、グループ作って濃い情報を共有し始めるような奴らが勝ち始めますよ。}

\question{最近のタイパーについてはどうでしょう。}

\answer{俺}{Twitter 効果で毎パソ勢とか歌謡\footnote{歌謡タイピング劇場。ハンゲーム内のタイピングゲーム。}勢とかがタイプウェルやったりしてて良いですよね。しかもめちゃくちゃ速い人がいるっていう。ちょっとやり込んで常用 ZH とか。あと我流運指勢が目立ちますよね。RGB (o-ck) 氏の影響で我流イコール安定しないみたいなイメージになっちゃってますけど、そうでもない人も多かったり。やっぱ同じ指を連続して使わなくて済むからネックになるところが少ないんで、そのままなんとなく打ってるだけで ZG くらいまで行っちゃうんじゃっていう。最適化勢ってあんまり研究って感じじゃなくて、本能的に打ってる感じがするんすよね。考えるんじゃなくてこっちのが気持ちよく速く打てるからそうするみたいな。今はまだ later レベルの人はいないですけど、指多く使って最適化もバリバリしてってスタイルの方はいるんで、あと数年すると later クラスに化けるんじゃねーかなと思って見てます。当時の競技人口であのランキングの密度だったから凡人の俺も一応トップになれましたけど、いずれあれも「センスがなくてそれなりにやり込み続けた人の平均」くらいになるんじゃねーかなと。}

\question{かなりギャラリー的な話になってましたが、少し戻して……タイピングの要素として、レイテンシ\footnote{ワードを認識してから、最初の一打鍵をするまでの時間。}についてはどうでしょう。}

\answer{俺}{俺レイテンシは調子良いときだと 300ms\footnote{ms はミリ秒。1000ms = 1 秒。} から 350ms ぐらいっす。今やったら多分 350ms から 400ms くらいですね。なんか意識して鍛えだすと、30ms くらい縮まってきて、あとなぜか調子良いときあるじゃないですか、なぜか速いとき。そうすると 300ms 近く行くんですよね。エタイやり込んでた時期はそこらへん伸ばそうとしてた。で、タイプウェルは遅くなってたっていう(笑)}

\question{エタイ (e-typing) も今でこそ抜かれてしまいましたけど、元気ワードの 817pt、当時の史上最高記録を出されてましたね。}

\answer{俺}{あれ確か文字数はそんなになかったんですよ。ある程度の速度でガーッって打てたら来たんですよね。あの時は初速もかなりあったんすよ。あきうめの動画とかで、レイテンシ 366ms とか出てたじゃないですか。あれよりは速いレイテンシでぽんぽん打てていて。だからあきうめの初速単体自体はそこまでめちゃくちゃすごいという風には捉えてないっすよね。later の打鍵力が SS ならあきうめの初速は S とかで。ほんとなんかマジで鍛えたら 250ms とかぽんぽん出せるやつ出てるはずなんすよ。}

\question{特に意識的な練習をしてないのに、なぜかはじめからレイテンシ小さい人というのがいますよね。}

\answer{俺}{そう。そういう人が動作の方を省略していったらもっと速くなりますよ。若い人って初速伸びやすいイメージが。せっかくだし今ちょっと八重タイ\footnote{八重タイピング。e-typing腕試しモードのクローンソフト。}やってみますか。初速狙いで。}

\question{(笑)}

\answer{俺}{1 ワード目ってちょっと初速出にくいですよね。……はい、今 1 ワード目が 414ms で、2 ワード目が 342ms でしたね。}

\question{342ms ってめちゃくちゃ速いですね。やっぱ一打鍵目の文字が関係あると思うんですけど、\key{K}とかが速いんですかね。}

\answer{俺}{K ……ってよりは\key{S}が速いですかね。\key{K}は(同指異鍵の)「き」があるせいで認識が混ざっちゃうんじゃないですか。\key{S}だとそれがないんで。やっぱ八重タイは良いっすね、これユーザ全員合わせると累計で 10 万回とかやられてるんじゃないですか。やっぱ rkpm とかレイテンシとか、指標を出してもらえると、これがあるからこれ伸ばそうとかいう意識ができますからね。}

\question{光栄です。やはり公式 e-typing の表示だけでは何をしていいものか迷いますもんね。しかしそう考えるとその時代にあきうめ氏のあの記録があるというのは、やはり驚きます。}

\answer{俺}{それは……父親がいましたからね。父・信仁さんがいなかったら多分、あきうめ凡俗的なトップタイパーに留まってましたよ。トップだった頃の俺より一枚格上だよとかその程度だったと思う。そのカリスマ性とかが。競馬で言えばあきうめは三冠馬じゃないですか。それが有馬記念とか宝塚記念一回勝ったくらいになっちゃう。もしくは二冠馬くらい。}

\question{やはりあきうめ氏には一目置かれてるんですね。}

\answer{俺}{っていうか、ぶっちゃけあきうめに勝ったって思ってる時期なんてないですからね。常に俺より上だろっていう。(e-typing 腕試しの)ポイントは超えたけど、あきうめがやり込んだら絶対抜かれるだろうっていう。あの最盛期には長文速度維持力的なところでは勝ってると思ったんですけど、全体的に見たらやはり負けるか、同等かなと。で正確性とか反応速度とかゲーム的な要素が出てくると、これはもう勝てんなと。あと、そういう純粋な能力もすごいっすけど、本当は競争の中でめっちゃ伸びるタイプなはずなんですよ。競争する相手がいなかったから、あの程度なだけ。いたらもっともっと上行ってましたよね。多分当時のエタイに今のひろりんご (later) さんがいて、820pt とか出してたら、850pt とか出し返してましたよね。あきうめですから! だからあきうめに一目置いてたんですよ。一目どころじゃない、絶対抜き返してくるっていう確信。}

\question{伝説的になっていますね。}

\answer{俺}{(俺氏があきうめ氏の e-typing を抜くことで)絶対にやり返してくるからって、期待してたんですけど……遅すぎましたね、俺は。}

\question{とても残念そうですね。当時の彼について行けたのは、本当にごく一部の人だけでしたからね。}

\answer{俺}{まあ勃起さんとか俺とか魔界タイパーも、ちょっとカスった程度でしたっていう(笑)}

\question{そろそろ締めようかと思います。何か最後にメッセージを頂けますか。}

\answer{俺}{メッセージって言っても、言いたいことあったらブログで書いてますから(笑)うーん(笑)}

\question{何か面白おかしい方向とかでいいんじゃないですか。}

\answer{俺}{じゃあ……これで!「タイピングは文字を打つものであり、文字は言葉を紡ぐものです。言葉とは――タイピングとは、コミュニケーションの手段なのです。『vl D ipftoi K mtms!』」}

\question{ありがとうございました。}
